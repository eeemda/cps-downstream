\documentclass[hidelinks, 12pt, titlepage]{article}

\frenchspacing
\usepackage{authblk}
\usepackage{times}
\usepackage{mathdesign}
\usepackage[T1]{fontenc}
\usepackage[protrusion=true]{microtype} %Allows for the protusion of punctuation marks to allow for cleaner justification of text
\usepackage{fullpage}
\usepackage{setspace}

\usepackage{natbib}
\setlength\bibsep{0cm}

\usepackage[fleqn]{amsmath}

\usepackage{graphicx}
\usepackage{array}
\usepackage{rotating}
\usepackage{longtable}
\usepackage[flushleft]{threeparttable}
\usepackage{threeparttablex}
\usepackage{booktabs}
\usepackage{epsfig}
\usepackage{epstopdf}
\usepackage{lscape}
\usepackage{multirow}
\usepackage{pdflscape}
\usepackage{xcolor}
\usepackage{subfigure}
\usepackage{tfrupee}
\usepackage[stable]{footmisc}
\usepackage[hyphens]{url}
\usepackage{hyperref}
\usepackage{doi}
\usepackage{nameref}
\hypersetup{
    colorlinks,
    linkcolor={red!80!black},
    citecolor={blue!50!black},
    urlcolor={blue!80!black},
    breaklinks=true
}
\urlstyle{same}
\usepackage{fancyhdr}
\usepackage{color}
\usepackage[font=small,labelfont=bf]{caption}
\usepackage{floatrow}
\floatsetup[table]{font=scriptsize, capposition=top}
\floatsetup[figure]{capposition=top}

% Allow line breaks with \\ in specialcells
	\newcommand{\specialcell}[2][c]{%
	\begin{tabular}[#1]{@{}c@{}}#2\end{tabular}}
%Reference: http://www.jwe.cc/2012/03/stata-latex-tables-estout/ Accessed 14-09-28

\newcommand{\ed}[1]{\color{blue}[Re-check: #1]\color{black}}

\title{The Lessons Private Schools Teach: Using a Field Experiment to Understand the Effects of Private Services on Political Behavior\thanks{I would like to thank Alexandra Cirone, Aniruddha Jairam, Ashutosh Varshney, Bryce Millett Steinberg, Devesh Kapur, Guy Grossman, Milan Vaishnav, Nina Brooks, Tulia Falleti, Ujju Aggarwal, workshop participants at the University of Pennsylvania, New York University, University of Wisconsin-Madison, Harvard University, the Brown-Harvard-M.I.T. Joint Seminar on South Asian Politics and the 2014 American Political Science Annual Meeting, as well as three anonymous reviewers and the editors at \emph{Comparative Political Studies} for insightful comments that have helped improved the paper immeasurably.  Financial support for research and writing was provided by the University of Pennsylvania's Penfield Dissertation Research Fellowship, the Browne Center's Dissertation Research Grant, the American Institute for Indian Studies Junior Fellowship, and the National Academy for Education/Spencer Dissertation Fellowship.  I am forever indebted to M. Rajani and the field coordinators from the Azim Premji Foundation for help implementing and conducting the survey.  This project could never have happened without the support and input of Karthik Muralidharan.  I also thank the survey respondents for their infinite patience.  The survey and interviews described in this paper were approved by the University of Pennsylvania's Institutional Review Board, Protocol \#818751. Replication materials and code can be found at the \href{https://doi.org/10.7910/DVN/TDUVCJ}{\emph{Comparative Political Studies} dataverse} or the \href{https://github.com/eeemda/cps-downstream}{author's personal github} \citep{Davies2022a}.}}

\author{Emmerich Davies\footnote{Assistant Professor, Harvard Graduate School of Education. Email: \href{mailto:emmerich_davies_escobar@gse.harvard.edu}{emmerich\_davies\_escobar@gse.harvard.edu}}}

\date{\today\footnote{Conditionally accepted at \emph{Comparative Political Studies}}}

\begin{document}

\pagenumbering{gobble}

\maketitle

\abstract{Government services act as important sites of political socialization.  Through interactions with the state individuals learn lessons about their value as citizens, form preferences over government services, and understand the value of political participation.  What happens when the private sector replaces government provision of basic services?  I explore this question in the context of a randomized private school voucher experiment in India.  Based on an original household survey of 1,200 households, and semi-structured interviews conducted five years after the voucher lottery, I find that voucher winning households hold stronger market-oriented beliefs.  However, voucher winning households show little difference in political participation on most measures of political participation.  I argue that these results are driven by a greater comfort with private providers as permanent economic actors.  This suggests exposure to different economic actors, in the form of private schools, have the potential to change political preferences.}\\[24pt]

\begin{center}
	\textbf{Word Count: 11,095}
\end{center}

\clearpage
\pagenumbering{arabic}
\setcounter{page}{1}
	
\doublespacing

	Government services have been found to act as important sites of political socialization.  Through interactions with institutions and functionaries of the state, individuals learn important lessons about their worth as citizens and the functioning of democracy, form preferences over government services, and understand the value of political participation.  At the same time, private actors are providing a greater number of basic services across the developing world \citep{Cammett2011}.  What happens when governments no longer provide basic services and are replaced by private actors?  Do private actors break the link to this political socialization process?  And does exit from state-provided shape political preferences in politically consequential ways?  Scholars have feared that as states cease to provide services and private actors emerge to fill the vacuum, citizens will become politically ambivalent as they no longer require the state to provide services \citep{Hirschman1970}.

	I explore state exit in India, where citizens have increasingly turned to private organizations for basic services.  This paper takes the case of education, where over 40 percent of households send their children to private schools \citep{ASER2019}, and asks what happens to political attitudes and behaviors when citizens exit state services for the private sector.  I argue that state exit can have two types of effects on politics: on how citizens participate in politics as a result of different material and social positions, and on how citizens interpret politics as a result of different experiences \citep{Pierson1993}.  Despite the importance of these questions, making conclusive causal claims is difficult as the growth of private services tends to be historically contingent and highly endogenous to political and economic variables.  To overcome these problems, I leverage a randomized private school voucher program and provide causal evidence on the effects of private services on political socialization.  I find that access to private services does not depoliticize citizens.  Instead, they shape economic preferences by making citizens more comfortable with a greater role for the private sector in service provision.

	Specifically, I leverage a randomized school voucher lottery to understand the political consequences of state exit.  In 2008, households across five districts of Andhra Pradesh, a large state in South India, were offered the opportunity to enter a private school voucher lottery, and lottery winners could send their child to a private elementary school for four years. I returned to these households five years later and employed a mixed methods approach, including an original survey of 1,202 households that entered the lottery, and 30 semi-structured interviews with program participants and education bureaucrats in the state to test the effects of private services on political outcomes.  I find that households that sent their children to private schools become more comfortable with paying out of pocket for other services that are currently provided by the government, which I take as evidence of increasing comfort with the private sector.  Political participation --- measured either by voting, a number of more costly partisan actions such as canvassing for a political party, or non-partisan local participation --- shows few differences between treatment and control groups on most measures.  I argue that while exit from government services has an effect on mass publics, it is on economic preferences, and not political behavior.

	I explore three potential mechanisms for what could be driving results: better objective or subjective quality of private schools, the breaking of ties between citizens who attend private schools and the Indian bureaucracy, and greater comfort with the private sector as a result of increasing ties to the private sector.  Evidence suggests that the results are driven by greater comfort with the private sector, which I call ``legibility''.  Voucher winners report that they came to understand the functioning of the private sector through access to private schools, and that they had greater faith in the private sector continuing to operate in low-income communities.  I find no evidence for private schools being of better quality than government schools or that moving children to private schools severed ties between citizens and the Indian bureaucracy.  Private schools were of equivalent quality as government schools, and respondents in the treatment group reported similar levels of contact with government bureaucrats even after exiting to the private sector.

	These findings are important not only for what they tell us about the Indian case, but what they reveal about service provision more generally.  While social science has traditionally assumed that the state is the primary provider of basic services \citep{Post2018}, the private sector is increasingly an important service provider both in the U.S. \citep{Morgan2011a}, and in low- and middle-income countries \citep{Cammett2011}.  This paper adds to a growing body of work that suggests we should take the political effects of religious \citep{Clark2004a,Thachil2011,Thachil2014a}, non-governmental organizations \citep{Boulding2014,Bratton1989}, and private organizations seriously \citep{Lerman2013}.  Much of this work has focused on the strategic use of service provision by political parties to win votes \citep{Cammett2010,Clark2004a,Thachil2014,Thachil2014a}.  Here I argue and show that even private organizations without explicit political goals have important political effects by changing a recipient's perceptions of their social world \citep{Lerman2013}.

    The case of education in low- and middle-income countries also merits special attention. I provide causal evidence of the effects of schools on mass publics.  The literature on the effects of education in political science has largely been focused on the impact of increasing \emph{access} on students political participation \citep{Croke2016,Dee2004, Marshall2016, Marshall2019,Mettler2005,Rose2018,Wantchekon2013}. I show that schools impact not only the direct beneficiaries -- students -- but also a larger group of citizens that form opinions and take action on politics as a result of their interaction with schools. Education is a part of the larger welfare state and as such structures individual level beliefs about government beyond students \citep{Kumlin2004}.  As a result of ``Education For All'' reforms, many low- and middle-income countries have greatly expanded access to education \citep{Bold2015,Pritchett2013a}, yet private providers have thrived \citep{Post2018}.  This has happened at the same time as state retrenchment in low- and middle-income countries have led to the rise of subsidiary service provision arrangements.  Most of our evidence on these effects come out of high-income countries \citep{Cook2020,DeAngelis2019,Dill2009,Kingsbury2019}.  Given the increasing growth of charter and private schools in high-income countries this is not surprising but still represents an unfortunate omission as many low- and middle-income countries such as Liberia \citep{Romero2020}, India, and Kenya and Tanzania \citep{Bold2015,Lucas2012} have seen a rapid growth of private schools.

	Methodologically, I rely on a downstream or secondary experiment to uncover causal effects of interests to political scientists \citep{Baldwin2011}. I take advantage of an existing voucher experiment implemented by a separate organization and return to respondents asking a different set of questions than the original implementers.  With the increasing costs of data collection and fieldwork in comparative politics \citep{Hsueh2014}, and the causal revolution in the discipline \citep{Humphreys2020}, this paper provides an example of a productive avenue for scholars to leverage already existing randomized control trials to ask questions that speak to questions in political science.  Theoretically, I explore learning effects as a mechanism for institutional change through an exogenous shock and later endogenous feedback effects \citep{Pierson2000b}.  This provides an example of how to integrate literatures on institutional change and stability, and experimental methods, two literatures that have largely remained separate \citep{Steinmo2015}.  

\section*{Theoretical Expectations\label{section:litreview}} 

	How should we expect access to private services to change public opinion and behavior?  The growth of private services could have an impact on behavior and opinions through \emph{selection} or \emph{treatment}.  Albert \cite{Hirschman1970} suggested that citizens \emph{select} into private services, and that those most likely to exit government service provision are also those most likely to exercise their ``voice'' to hold public sector providers accountable.  Through exit, individuals discipline a service provider for a decline in quality by no longer using the service, while by using their ``voice,'' individuals express their displeasure directly to the service provider.  The high levels of state exit in India might help partly explain that education has rarely been an important electoral issue in India \citep{Ansell2010,Rudolph1972,Rudra2002,Weiner1990}.

    This idea has found empirical support in the United States and sub-Saharan Africa, where the use of private services has reduced the exercise of citizenship \citep{Abernathy2005,Cook2020,MacLean2011}, although findings are more indeterminate overall in the United States \citep{DeAngelis2019, Dill2009, Kingsbury2019}.  State retrenchment after market-oriented reforms in sub-Saharan Africa has meant that citizens have turned to private service providers for basic services such as health and education, leading to decreased political participation as citizens no longer needed to mobilize publicly for quality service provision \citep{MacLean2011}.  Through a different mechanism, Jaimie \cite{Bleck2015} finds that sending children to government schools in Mali results in greater levels of political participation from parents.  Families with students in government schools use these students as ``linguistic brokers'' to overcome linguistic barriers to greater political participation. In response to the declining quality of public services or continued access, citizens exercised voice and mobilized to demand better services from government officials through increased participation and ties to the state, while those that stop using public services exercised less voice.

	In this paper, I focus on the \emph{treatment} effects of private services. Education service provision provides an important test case for how public policy affects mass opinion as education service provision can directly change an individual's capacity for political action as well as indirectly framing political problems in new ways, impacting how individuals exercise both exit and voice.  I argue there are two potential mechanisms through which school privatization could impact political opinions and behaviors and therefore how exit and voice are used.  ``Resource effects'' occur as, ``[new policies] provide resources and incentives to individuals that profoundly influence crucial life choices: what kind of job to take, when to retire or take time off from the paid labor force, how to organize and divide household tasks such as child rearing,'' \citep[606]{Pierson1993}.  New policies also influence ``the manner in which social actors make sense of their environment,'' or what Pierson calls ``interpretive effects'' \citep[610-1]{Pierson1993}.

	The material effects of policies have been found to have a broad range of effects on political behavior.  The G.I. Bill in the United States encouraged returning soldiers to go to university and increased political participation \citep{Mettler2005}.  Senior citizens became politically active to defend their existing material benefits \citep{Campbell2003,Lynch2006}.  In India specifically, schools are often the first point of contact citizens have with the formal state \citep{Corbridge2005}, deliver the midday meal scheme \citep{Dreze2001}, and serve as voting locations \citep{Neggers2018}.  Through legally mandated forums, known as School Management Committees (SMCs), parents are asked to be involved in the governance of government schools, including the spending of budgets and the hiring of teachers.  Government teachers are also often the most educated members of their community, and engage in a number of non-teaching activities --- such as election monitoring and conducting the decennial census --- that make them highly visible in their communities \citep{Beteille2015}.  Teachers unions are powerful political constituencies in their own right, and frequently lobby politicians and voters to act in their interests \citep{Kingdon2009a}.  My own data suggests that upwards of 85 percent of government teachers served as either election monitors or census enumerators in the past year in my sample villages (see Table \ref{table:summarystatistics}).

	In addition to the role that teachers and schools play in politics, there is frequent political interference from local politicians.  Pressure is placed on District Education Officers (DEOs) to transfer particular teachers as punishment for poor performance, despite this being outside their responsibilities.\footnote{Interview with M. Somi Reddy, District Education Officer Ranga Reddy District, September 2013 and \cite{Beteille2015}.}  An interview with a parent suggested that parents would approach Members of the Legislative Assembly (MLAs) and lower-level elected politicians if they had concerns in their children's schools, instead of the education bureaucracy.\footnote{Interview with voucher household in Visakhapatnam District, November 2013.}  One survey respondent gave examples of how in the run-up to the 2014 legislative assembly election, the local MLA had begun to visit their village more frequently and had recently promised \rupee 100,000 (approximately \$1,600 at the time of field work) to the village primary school to build a wall around the school and provide board games for the children at the school.\footnote{Interview with voucher household in Medak District, November 2013.}

    Finally, while many interactions between citizens and the state are mediated by intermediaries \citep{Auerbach2016, Berenschot2010,Krishna2007}, and the Indian state has also been described as ``omnipresent but feeble'' \citep[6]{Kohli1990} -- it is ever present in the lives of its citizens, especially the poor.  The expansion of education over the last three decades has mean that, at the very least, citizens can expect a school and teacher to be present \citep{GovernmentofIndia2009}.  The high levels of rural decentralization \citep{Auerbach2020a,Kruks-Wisner2018}, and widespread social protection \citep{Kapur2013a}, has meant a greater presence in the social and economic lives of rural citizens.

	Government schools act as a site of interaction between citizens and the Indian state.  By providing opportunities and grievances over which to mobilize over the provision of club goods, bringing voters into contact with the state through the provision of services such as the midday meal, contact with teachers as front-line functionaries of the state, and contact with political parties and electoral politics through political interference, government schools provide a number of paths through which individuals can become involved in politics.  By contrast, providing a way to exit these relationships through privatization and increasing contact with actors in the non-state sector can change the relationship between citizens, the Indian state, and partisan politics.  This provides us with H$_{1}$:

	\begin{itemize}
		\item H$_{1}$: Private school voucher lottery winners will be less likely to participate in political forums.
	\end{itemize}

	Interpretive effects provide a second channel through which exit from the state can impact politics.  Political learning in the public sector provide individuals with greater knowledge ``in the operation of complex systems [that lead] to higher returns from continuing use,'' \citep[254]{Pierson2000b}. Individuals learn about politics through policies and interactions with representatives of the state.  These experiences provide citizens with examples of how the state views them and how they can better engage in repeated interactions with the state that then have spillovers to other domains \citep{Pierson1993, Soss1999}. Early experiences with representatives and institutions of the state shape knowledge and incentives and knowledge on how to later engage with the state and exhibit increasing returns from continued use.

Similarly, experiences with the private sector have the power to shape political attitudes and behaviors independent of any material change in the relationship between states, markets, and households.  Experiences with private services that are \emph{new} and then \emph{sustained} should also exhibit increasing returns as formerly opaque and complex systems become more ``legible''.\footnote{This is different from James Scott's \citeyearpar{Scott1998} legibility whereby the state makes its \emph{citizens} more legible.  In this instance, it is the state or private sector that becomes more legible to ordinary citizens.}  For example, access to financial markets have shifted political beliefs in the United Kingdom by increasing familiarity with markets \citep{Margalit2020}.  Even if individuals do not make money in the stock market, access to the stock market provide new and sustained experiences for participants that led to greater familiarity and decreasing distrust of financial markets.  The privatization of state-owned enterprises in Czechia encouraged citizens to acquire information about how they operated and led to greater engagement with formerly state-owned private enterprises \citep{Earle2003}.  Markets in health care in India made health care recipients ``energetic and entrepreneurial in shopping around,'' \citep[133]{Jeffery2008}.

New experiences after market-oriented reforms introduce individuals to new market institutions and experiences that, with repetition, allow them to understand and use these institutions more effectively.  In the case in this paper, citizens learn by understanding how the private sector, and one service in particular, works as these were new experiences for most households and provide a direct way through which to experience the functioning of the private sector.  Two design features of the voucher encouraged this learning. First, households were only eligible to enter the program if they did not have any children in private schools already, and evidence from other states in India suggests that while households often rely on private tutors \citep{ASER2019},\footnote{It is important to note that most private tutors are public sector teachers \citep{Jayachandran2013}, blurring the distinction between public and private for tutoring services.} very few households from marginalized groups send their children to private schools \citep{Romero2022}.  Most recipients in this voucher experiment lived in rural or peri-urban areas with little access to elite private schools.

A second design feature that encouraged learning and increasing returns is that the voucher was provided for four years, meaning that households had sustained experiences with private schools.  While many voucher programs in the United States are administered by the state, the voucher program in this paper was implemented by the Azim Premji Foundation, a well known NGO in South India, and would have moved households away from government services, making this case analogous to that of \cite{Carlson2016}.  Field observations and administrative data suggest that most private schools in the area where the vouchers were offered were relatively new, with an average age of four to eight years at the beginning of the voucher lottery.  Private schools were often at the outskirts of villages and did not have dedicated school buildings, often renting space in existing houses.  This was in contrast to government schools that were often in the center of villages or near major thoroughfares, with dedicated land and buildings for the schools.  In this case, experiences with a new service provider can open the black box of the service, but depending on the program design, voucher and similar programs can highlight the importance of state regulation in the performance of schools \citep{Fleming2014}, or, if responsibility for provision is unclear, lead to greater confusion of who has ultimate responsibility for the benefits and failures of particular service providers \citep{Carlson2016}.

These effects are also independent of the quality of the service.  For experiences such as education (here) or the stock market (as in \cite{Margalit2020}, quality is often only revealed in the long run. The understanding gained from repeated use, however, materializes quickly. An experience on the first day of school or selling your first share provides lessons for the second day of school or second sale.

Citizens in India experience many government policies through schools and have experiences that can structure their perception of the value of political participation that are no different from experiences in high-income welfare states \citep{Kumlin2004}.  The high legibility case is analogous to the experiences that voucher winners underwent in the voucher lottery.  This suggests that newfound access to private services introduce individuals and households to a new set of social actors, and makes certain parts of the private sector more legible to households.  This legibility should make households more comfortable with market actors leading to H$_{2}$:

\begin{itemize}
\item H$_{2}$ Private school voucher lottery winners should hold stronger market-oriented economic preferences.
\end{itemize}

Before turning to data collection and results, I explore the Indian context in the next section and introduce the field site in which I conducted fieldwork.

\section*{Private Services in India and the Andhra Pradesh School Choice Experiment\label{section:context}\footnote{At the time of fieldwork in late 2013, the states of Andhra Pradesh and Telangana were part of united Andhra Pradesh that was later divided following the 2014 elections.  I refer to the state of Andhra Pradesh in this paper although two of the districts, Medak and Nizamabad, are now in Telangana.}}

	The private sector, including religious and non-profit organizations, have expanded rapidly across India \citep{Kingdon2017}.  While exact data is hard to come across, it is estimated that approximately 50 percent of urban and 20 percent of rural primary schooled aged children attend private schools \citep[6]{Kingdon2017}.  Acknowledging this reality, the Indian Government passed The Right to Education (RTE) Act in 2009 \citep{GovernmentofIndia2009}. Section 12.1(c) of the Act specifies that private schools must accept 25 percent of their incoming class ``belonging to weaker sections and disadvantaged group in the neighbourhood and provide free and compulsory elementary education till its completion,'' \citep[5-6]{GovernmentofIndia2009}.  Some commentators have called the policy, ``India's civil rights moment,'' as it would bring disadvantaged groups in contact with the country's elites.\footnote{Interview with Menaka Guruswamy, Supreme Court Advocate, November 2012.}  The policy effectively serves as a private school voucher as the state compensates schools for children admitted under this clause to attend private schools \citep[6]{GovernmentofIndia2009}.  Unlike other voucher systems where households receive vouchers, the state reimburses private schools directly, so students and parents do not receive a transfer from the state.  At the same time, the policy has not been without controversy -- it was challenged in courts by an association of private schools,\footnote{Society for Unaided Private Schools of Rajasthan v. Union of India and Another, No. 95 of 2010.} and concerns remain that private schools discriminate against the children accepted into their schools under this policy \citep{Romero2022}.

	The Andhra Pradesh School Choice Experiment (from here on APSC), which this paper follows-up from, sought to provide evidence of the effects of the RTE on educational outcomes \citep[1018-19]{Muralidharan2015}.\footnote{Further details on the APSC are provided in Appendix Section \ref{appendix:apsc} or \cite{Muralidharan2015}.}  The original design relied on a two-stage randomization process.  In the first stage, villages were eligible if they had at least one private school as this would allow parents to choose a school for their children in their village.  All villages were then informed that they would be entered into a private school voucher lottery run by the Azim Premji Foundation, a well known education NGO in South India.  Villages were then randomized into treatment and control, creating a \emph{village-level} counterfactual where some villages would be eligible to receive private school vouchers and others would serve as the control group.  Next, households within treatment villages were then randomized in the second stage to receive vouchers, thereby creating a \emph{household-level} counterfactual.  Because of resource constraints, I only sampled from treatment \emph{villages}, so I was only able to explore individual and not community level effects, fully aware that those questions are also of interest to researchers in political science.\footnote{Treatment and control households were sampled \emph{within} treatment villages.}  In the larger experiment, 23 percent of \emph{all} students originally in public schools in treatment villages accepted a voucher and moved to a private school, and 8 percent of students in private schools were voucher children \citep[1026]{Muralidharan2015}.  While the original experiment was administered by an NGO and not the government, the larger design mimics what then became national government policy.

	The state of Andhra Pradesh represents an ideal location for this study as the state has tremendous amounts of cultural, linguistic, and religious diversity that reflects the larger diversity of the country.  The state is the furthest North of the South Indian states (see Figure \ref{fig:apmap} for its location in India as well as the specific survey districts) and has a significant Islamic colonial legacy in its Northern districts and a strong British colonial legacy in the East and South.  This history makes its Northern districts akin to states in India's Northern Hindi belt, while its Southern and Eastern districts more similar to India's Southern states.  I conducted fieldwork in five different districts across its three major regional areas, Telangana, Rayaleseema, and Coastal Andhra.  The three Southern districts in this study were part of Andhra State at independence prior to joining Hyderabad state after the States Reorganisation Act of 1956.  This is not dismiss Andhra Pradesh's unique policy making environment.  Chief Ministers in Andhra Pradesh have led the way in attracting foreign capital \citep{Rudolph2001}, and in implementing technocratic forms of governance \citep{Bussell2012}, of which this policy experiment is another example.

	\begin{figure}[htbp]
		\caption{Survey Districts in Andhra Pradesh\label{fig:apmap}}
		\centering
		\begin{minipage}{5.5in}
			\includegraphics[width=5.5in, keepaspectratio=true]{Data/output/figures/mapdistricts.pdf}
			\footnotesize
			\emph{Notes:} The state of Andhra Pradesh is shaded in light gray, while districts in which the intervention and subsequent survey were conducted are in dark gray.
		\end{minipage}
	\end{figure}

	The state has also led attempts in understanding the impact of private schools on student learning outcomes.  The survey I conducted followed a series of large scale experiments between the Government of Andhra Pradesh, the Azim Premji Foundation, and researchers from Harvard University, the World Bank, and the University of California, San Diego (see \cite{Muralidharan2011,Muralidharan2011a,Muralidharan2015}).  The voucher experiment was explicitly designed to mimic Section 12.1(c) of the Right to Education Act.  This provided a particularly conducive research environment in which to conduct research on the effects of the private sector.	Using the original randomization list provided by \cite{Muralidharan2015}, I returned to households five years after they entered a private school voucher lottery and one year after their children had completed primary school and the voucher lottery experiment and administered an entirely new survey instrument.

\section*{Data and Methods\label{section:dataandmethods}}

	To test the hypotheses laid out above I rely on an original household survey of 1,202 households collected between September and December 2013 in five districts of Andhra Pradesh.  The survey was conducted by a team of 11 surveyors administering an original in-person survey.  I retained the same survey team as used in the original APSC experiment to maintain the trust of respondents given the different set of survey questions asked here.  Between August and December 2013, I also conducted 33 semi-structured interviews with households that entered the lottery.\footnote{Further details on the interview sample is provided in Appendix \ref{appendix:interviews}.}  The survey sample was randomly drawn from the APSC Experiment. As the original experiment was stratified by district, this newer sample reflects this stratification.   The voucher was provided for four years of primary education that covered Standard 1 through Standard 4 (equivalent to 1st through 4th Grade), and I surveyed households when their scholarship child had finished 5th Grade and was entering secondary school.

	\subsection*{Operationalizing the Key Outcome Variables}

		This paper is interested in the effect of private schools on two broad outcomes of interests: 1. Political participation, and 2. Market-oriented beliefs.  I operationalize these latent outcomes by grouping several observable variables into weighted indices.\footnote{\cite{Anderson2008} recommends constructing the indices by taking the weighted mean of the standardized means of the individual variables that compose the index.  The weights are used to maximize the amount of information captured by the index by giving greater weight to uncorrelated variables and is the inverse of the covariance matrix.  By doing this, I increase statistical power while being robust to over-testing as I am only testing one outcome instead of a series of measures.  Using an index ensures that researchers do not cherry-pick results that might be significant by chance and misinterpret the importance of individual components of the index.  For transparency, I report the results of each individual component of the index as well as the full index. An additional benefit of using the index is that two index components of an index that would involve trade-offs in terms of time or money such as participation in two different associational groups as in the Political Behavior index, or how much a respondent would want to spend, such as the Valuation of Public Services Index, will give greater weight to anomalous positive outcomes.}  In the following section, I describe the individual components of each index.

	\subsubsection*{Political Behavior}

		\emph{Partisan Political Participation} is an index composed of four variables: whether a household member is a member of a political party, whether they attended a political meeting over the past year, whether they canvassed for a political party over the past year, and whether they distributed leaflets for a political party over the past year.

		\emph{Non-Partisan Political Participation} is an index composed of four variables: whether a household member is a member of a caste association, whether a household member is a member of a cooperative or labor union, whether a household member is a member of a self-help group (SHG), and whether a household member attended a \emph{gram sabha} (village government) meeting in the past six months.

	\subsubsection*{Market-Oriented Beliefs}

		Turning to measures of market-oriented beliefs, I attempt to measure market-oriented beliefs through stated and revealed preferences for the private sector, and through a measure of how much money households would be willing to spend out of pocket to receive services through the market.  I do this through two indices, \emph{Preference for Private Services} and \emph{Valuation of Public Services}.

		\emph{Preference for Private Services} is an index composed of six variables: whether respondents would prefer a job in the private sector or with the government, whether respondents would prefer the private \emph{financing} of services like health and education, whether respondents would prefer the private \emph{provision} of services like health and education, whether respondents continued to send their voucher child to a private school \emph{after} the private school voucher lottery was finished, the number of children in the household in private schools excluding the voucher child,\footnote{It is important to note that 45\% of private primary schools in the five survey districts only offered classes up to Class 5, and 90\% offered classes up to Class 7, meaning that most parents and students would either have already switched or would have to switch to a \emph{different} private school within a year of when the survey was field (Author's calculations from the District Information System for Education (DISE) School Report Cards for the 2013-14 academic year.)} and whether respondents go to a private health care provider if a household member falls sick.

		\emph{Valuation of Public Services} is an index composed of a respondent's valuation of two government services currently provided in-kind: publicly provided education and food rations.  I presented respondents with a hypothetical scenario in which households could either receive a cash transfer from the government of a certain value to purchase services on the private market or continue to receive the service from the government in-kind, attempting to reveal a respondent's ``valuation'' of two important government services.  For example, in the case of education I gave respondents the hypothetical option of either receiving a school voucher and being able to shop for a school for their children as they pleased \emph{or} using the government school system.  The value elicited provides the value at which respondent's would be indifferent between receiving the value in-kind from the government or shopping for similar quality goods on the private market.  I take this answer to represent the value respondent's attached to government services --- a higher response suggests it would take a higher value voucher to convince respondents to abandon government services for privately provided services.

		For school vouchers [food rations], starting with the amount of \rupee3,000 per year [\rupee200 per month], surveyors asked respondents whether they would prefer to receive government provided education in-kind [their ration that month] or the specified amount in cash.  The amount was increased in \rupee500 [\rupee50] increments until either the respondent said they would rather receive that amount in cash \emph{or} a maximum offer of \rupee10,000 [\rupee1,000] was reached.\footnote{For school vouchers, surveyors also informed households how much that amount was worth per month to facilitate calculating monthly expenses.}  If the respondent did not accept any offer below \rupee10,000 [\rupee1,000], the surveyor asked for the minimum amount the respondent would be willing to accept instead of government provided education directly [subsidized rations at government ration shops].

		While uncommon in political science, this method of eliciting a person's valuation of a good is more common in the development economics literature \citep{Kremer2011}.  \cite{Kremer2011} caution against over interpreting the results from these hypothetical scenarios as valuations in hypothetical are often larger than through actual expenditures and revealed preferences.  To validate these results, I include both the hypotheticals discussed here, as well as real preferences discussed in the preference for private services index.

		I suggest that their valuation of public services through this measure represents a respondent's relative preference for market services.  I pick these two services as they are two services that have a potential ``price'' for their provision that a respondent could understand and calculate, as well as services for which there is policy debate as to the best way to provide them in India, with the use of vouchers so individuals can access the markets a realistic proposal \citep{Kapur2008}.  The two services represent clear subsidies from the state to individuals and removing them would force households to bear expenses out of pocket.  The choice I forced households to make is between having the respective service be subsidized by the state or shopping for that particular service on the open market.

	Table \ref{table:balance} presents a difference of means between treatment and control households for baseline covariates and time-invariant variables collected at endline.  The first six variables were collected at baseline and also presented for the full sample in \cite{Muralidharan2015}.\footnote{The math test was administered to a smaller sample of respondents at baseline accounting for the lower number of households for which results are reported.}  The next five variables --- gender, age, religion, and prior vote choice --- are time or treatment invariant.  The random assignment by lottery and subsequent random sampling ensured that there was covariate balance between treatment and control households on observables in the downstream sample.

	\begin{table}
		\begin{threeparttable}
			\caption{Balance Tests Between Treatment and Control Groups\label{table:balance}}
			\centering
			\input{Data/output/tables/balancetest.tex}
			\begin{tablenotes}
				\item \emph{Notes}: Difference in means represents the mean differences between treatment and control households.  To combine multiple tests into a single test statistic, I estimate a Seemingly Unrelated Regression and perform a Chi-squared test for the hypothesis that the coefficients on having been offered a voucher are jointly equal to zero.  As the baseline math test was administered on a smaller sample of students, we perform two tests, first with the full sample of baseline characteristics for which we have data on 652 students, and a second test not including baseline math characteristics, for which we have complete data for 1,123 students.  The p-value from the first test is \input{Data/output/text/fullchi} and from the second test is \input{Data/output/text/reducedchi}
			\end{tablenotes}
		\end{threeparttable}
	\end{table}

	Summary statistics for the variables used in this paper are provided in Table \ref{table:summarystatistics}.  64 percent of households that were offered vouchers were still sending their children to private schools after two years, and 59 percent after four, although there was significant non-compliance in the control group, with 26 percent of households sending their children to private schools after two years and 24 percent after four.  Consistent with data on Andhra Pradesh \citep{Deininger2012}, membership in SHGs was the most common form of partisan or non-partisan political activity across the sample, with lower rates of participation in costly political activities such as canvassing and mobilizing voters.

	\begin{table}
		\tiny
		\caption{Summary Statistics\label{table:summarystatistics}}
		\input{Data/output/tables/summarystatistics.tex}
	\end{table}

	There was also differential attrition between treatment and control groups in the downstream sample, with 7 percent more households reached in the treatment rather than control groups.\footnote{Further analysis of threats to the validity of the design from differential attrition and the effects of differential attrition are discussed in Appendix \ref{appendix:attrition}.}

	\begin{table}
		\begin{threeparttable}
			\caption{Compliance Rate of Downstream Sample\label{table:compliance_downstream}}
			\centering
			\begin{tabular}{lrr}
\toprule
	&	\multicolumn{2}{c}{Offered Voucher} \\
\midrule
    & No    & Yes \\\cmidrule{2-3}
Total & 405   & 797 \\
	&	&	\\
\multicolumn{1}{l}{\multirow{2}[0]{*}{Accepted Voucher Offer}} & NA    & 606 \\
\multicolumn{1}{l}{} &       & (76\%) \\
	&	&	\\
\multicolumn{1}{l}{\multirow{2}[0]{*}{In Private School After Two Years}} & 99    & 487 \\
\multicolumn{1}{l}{} & (26\%) & (64\%) \\
	&	&	\\
\multicolumn{1}{l}{\multirow{2}[0]{*}{In Private School After Four Years}} & 93    & 457 \\
\multicolumn{1}{l}{} & (24\%) & (57\%) \\
\bottomrule
\end{tabular}
			\begin{tablenotes}
				\item \emph{Notes}: Percentages represent percentage of households in treatment and control groups that won the lottery, accepted the voucher offer, were sending their children to private schools after two years, and were sending their children to private school after four years.
			\end{tablenotes}
		\end{threeparttable}
	\end{table}

	I rely on \emph{accepting} the voucher offer as the estimator of choice, using a voucher offer as an instrument for accepting the voucher. This can be seen as one definition of the Treatment on the Treated (ToT) estimate.  This is the estimator that most closely reflects the impact of having exposure to the private sector of which we are most interested.  I estimate a two-stage least squares regression that uses the offer of a voucher to instrument for enrollment in a private school with district and surveyor fixed effects.\footnote{Models with a full set of baseline covariates are robust to inclusion of these controls.  These models are available upon request.  I also provide the ITT, and the Treatment on the Treated (ToT) on having accepted the voucher, enrolled their child in a private school, and stayed in a private school for two and four years in that school in the tables in Appendix Section \ref{appendix:fullresults}.  Results presented in the main body of the paper are robufst to these different specifications and are robust to inclusion of baseline covariates.}  The estimating equation takes the form:

\begin{equation}
Y_{i} = \beta\text{Enrolled in Private School}_{i} + X_{i} + D_{i} + \epsilon_{i}, \label{equation}
\end{equation}

Where $Y_{i}$ is the outcome of interest, our coefficient of interest is $\beta$ which is the estimate of $\text{Enrolled in Private School}$ which is instrumented by being randomly assigned the voucher treatment.  $X_{i}$ is a vector of controls including whether both parents reached grade 10, a household asset index at baseline, whether households are either Scheduled Caste (SC) or Scheduled Tribe (ST), whether the respondent of the survey was male, their age, and a dummy for Muslim households.  Standard errors are clustered at the school the child was in at baseline.  I now turn to results on the impact of accepting a private school voucher lottery on partisan and non-partisan participation to test $H_{1}$, and market-oriented preferences to test $H_{2}$.

\section*{Results\label{section:results}}

	\subsection*{Political Behavior}

		I present results in Figures \ref{fig:partisan} through \ref{fig:valuation}.  Figure \ref{fig:partisan} presents results for a series of partisan political activities.  Although voucher winning households show higher levels of partisan political participation, the effects are both small and statistically insignificant except for the distribution of political leaflets that is positive at the 90\% level.  Rates of partisan political activity are generally low, ranging from about three percent of respondents that claim they are a member of a political party, to a high of 32 percent claiming that they attended a political meeting, and having access to private schools does not change this engagement substantially.  Even though the survey was conducted during a highly politicized period, immediately following village elections and five to eight months before highly salient state and national elections, rates of partisan political engagement remain low and similar between treatment and control.  Voucher winning households are less than 10 percent of a standard deviation more likely to engage in partisan political activities than non-voucher households, with this effect increasing with the costliness of the activity.  These effects are also small substantively.  Taking the case of reporting membership in a political party, receiving a private school voucher shifted the probability of reporting they were a member of a political party from 3 percent in the control group to 3.2 in the treatment group.

		\begin{figure}[htbp]
			\caption{Partisan Political Participation\label{fig:partisan}}
			\centering
			\begin{minipage}{5.5in}
				\includegraphics[width=5.5in, keepaspectratio=true]{Data/output/figures/figurepartisan.pdf}
				\footnotesize
				\emph{Notes}: Each plot presents results of a regression of the dependent variable on the left axis, on enrolling a child in a private school, instrumented by the original voucher offer.  Solid lines represent 95\% confidence intervals and tick marks represent 90\% confidence intervals.  Effect sizes are standardized to be comparable across specifications.  All specifications include district fixed effects.  Robust standard errors are clustered at the school at baseline. Tables \ref{table:appendixpolpart} through \ref{table:appendixpoliticalleaflets} are the corresponding regression tables for these plots.
			\end{minipage}
		\end{figure}

		Turning to non-partisan forms of participation, Figure \ref{fig:nonpartisan} shows that in an index of non-partisan forms of political and associational participation, as well as most of the individual components of the index, we see no differences between voucher winners and losers.  Although political participation in India is consistently high, and higher among low-income households \citep{Banerjee2011a}, there were few differences in political participation between treatment and control households.  Receiving a private school voucher led to approximately one percent higher probability that a respondent was a member of a cooperative association. One result is of note: that of membership in self-help groups, largely women's organizations designed to extend credit, share risk, and engage women in other activities in labor and private markets.  Results on this measure are positive and significant, with a five percent higher probability that a household member was a member of an SHG in the treatment group. Work in other states of India have shown that these groups can mobilize women into politics by extending their social and political networks beyond the household \citep{Prillaman2022}.

		\begin{figure}[htbp]
			\caption{Non-Partisan Political Participation\label{fig:nonpartisan}}
			\centering
			\begin{minipage}{5.5in}
				\includegraphics[width=5.5in, keepaspectratio=true]{Data/output/figures/figurenonpartisan.pdf}
				\footnotesize
				\emph{Notes}: Each plot presents results of a regression of the dependent variable on the left axis, on enrolling a child in a private school, instrumented by the original voucher offer.  Solid lines represent 95\% confidence intervals and tick marks represent 90\% confidence intervals.  Effect sizes are standardized to be comparable across specifications.  All specifications include district fixed effects.  Robust standard errors are clustered at the school at baseline. Tables \ref{table:appendixnonpartisanpoliticalparticipation} through \ref{table:appendixgramsabha} are the corresponding regression tables for these plots.
			\end{minipage}
		\end{figure}

		Returning to my original hypothesis on political participation, I find little evidence that exit from public services decreases political participation.  On almost all measures of political behavior, partisan political participation, and electoral participation, treated households show no difference in participation relative to control households.  Households that exited to the private sector are just as likely to exercise their voice than households that remained in government schools.  Indeed, for some measures of political participation, exiting the public sector might even \emph{increase} political participation, an effect I return to later while discussing mechanisms and causal pathways.  To preview that discussion, the effect on political participation might be on the \emph{networks} households are exposed to, rather than partisan attitudes and behaviors.  I now turn to economic preferences to understand if the \emph{content} of political engagement does change, given that political participation was so high.

\subsection*{Economic Preferences}

Returning to hypothesis 2, I argued that exposure to the private sector through private schools has increased voucher recipients's comfort with the idea of the private sector.  In the first part of the argument, I present evidence that access to private schools has resulted in a shift in market-oriented beliefs in an index and entire set of indicators in Figure \ref{fig:privatisation} on preferences for private services.  The index itself is positive, significant, and robust to different specifications.  Having won a private school voucher increases both stated and revealed preferences for the private sector by between 0.1 and 0.25 standard deviations as shown in the first coefficient plotted in Figure \ref{fig:privatisation}.  In other words, this suggests that having access to private school vouchers results in between a 4-10 percent increase in the number of respondents that say they would prefer the private sector providing services such as health and education.  The second coefficient, which reports results on whether respondents would prefer a job in the private sector, presents the hardest test.  Jobs with the Indian state often represent a salaried income, guaranteed employment, and solid pension, while jobs in the private sector are often precarious and ephemeral. In this test of preference for the private sector, there is no difference between treatment and control groups.

\begin{figure}[htbp]
    \caption{Preference for Private Services\label{fig:privatisation}}
    \centering
    \begin{minipage}{5.5in}
        \includegraphics[width=5.5in, keepaspectratio=true]{Data/output/figures/figureprivatisation.pdf}
        \footnotesize
        \emph{Notes}: Each plot presents results of a regression of the dependent variable on the left axis, on enrolling a child in a private school, instrumented by the original voucher offer.  Solid lines represent 95\% confidence intervals and tick marks represent 90\% confidence intervals.  Effect sizes are standardized to be comparable across specifications.  All specifications include district fixed effects.  Robust standard errors are clustered at the school at baseline. Tables \ref{table:appendixprivateservices} through \ref{table:appendixpreferprivatedoctor} are the corresponding regression tables for these plots.
    \end{minipage}
\end{figure}

In the other components of the index, however, we see a much stronger relationship with receiving a private school voucher and preferences for private services.  Questions on whether a respondent would prefer the private provision of services currently provided publicly, the private \emph{financing} of services provided publicly, whether households continued to send their voucher child to private schools \emph{after} the voucher had expired, the percentage of children in the household in private schools, and whether households would use private health services when household members are sick all show a \emph{positive} relationship between receiving a private school voucher and preferences for private services.  Aside from the hard test case of a job in the private sector, having access to private services results in both stronger stated and revealed preferences for private services.

To unpack another set of plausible preference changes, Figure \ref{fig:valuation} presents the results of the valuation of public services measures.  Private school voucher winning households are more likely to express preferences for lower valuations of cash transfers and this result is both statistically significant and large in magnitude.  Voucher winning households are willing to accept approximately \rupee240 lower transfers for both school vouchers and food subsidies.  For school vouchers, this represents a difference of about 6 percent from the control mean,\footnote{Rural households in Andhra Pradesh that sent their children to private schools spent approximately \rupee 10,060 out of pocket (Author's calculations from the 71st Round of the National Sample Survey on Education from 2014). The mean valuation for private school vouchers of households that enrolled their children in private schools through the Andhra Pradesh School Choice experiment was approximately \rupee 10,280, suggesting that it allowed households to better evaluate the cost of private schools.} but for the food subsidy, this represents 25 percent of the accepted amount amongst control households.  Most importantly, \rupee200 is about 1 percent of a household's monthly income, a significant difference in out of pocket expenditures for services they currently receive at a highly subsidized price from the government.  Through access to private schools, voucher winning households are more likely to suggest that they would reduce the level of government subsidies on key public goods such as food subsidies and government private services and turn to the market for these services.  This suggests a process of socialization and learning has occurred within voucher winning households, making them more comfortable with the idea of the private sector providing basic services.

\begin{figure}[htbp]
    \caption{Valuation of Government Services\label{fig:valuation}}
    \centering
    \begin{minipage}{5.5in}
        \includegraphics[width=5.5in, keepaspectratio=true]{Data/output/figures/figurevaluation.pdf}
        \footnotesize
        \emph{Notes}: Each plot presents results of a regression of the dependent variable on the left axis, on enrolling a child in a private school, instrumented by the original voucher offer.  Solid lines represent 95\% confidence intervals and tick marks represent 90\% confidence intervals.  Effect sizes are standardized to be comparable across specifications.  All specifications include district fixed effects.  Robust standard errors are clustered at the school at baseline. Tables \ref{table:appendixvaluation} through \ref{table:appendixvaluationpds} are the corresponding regression tables for these plots.
    \end{minipage}
\end{figure}

Returning to Pierson's differentiation between resource and interpretive effects, my finding that respondents developed stronger market-oriented beliefs suggest space for interpretive effects, but little evidence of resource effects. Within interpretive effects, I also find stronger results for private services that rural residents have less contact with: in the 2011-12 round of the Indian Human Development Survey (IHDS-II), rural residents are most likely to use government schools, then the Public Distribution System (PDS), and finally private hospitals (89, 85, and 28 percent respectively).\footnote{Author's calculations from \cite{Desai2018} for questions on whether households sent any children to private schools, whether they used a private hospital the last time they had to take a family member to the hospital, and whether the household used the PDS.}  The smaller shift in preferences for private hospitals relative to food and education is consistent with the larger theory that suggests these experiences need to be new to have an effect.  In the next section I suggest that this effect is a result of changed experiences with service providers themselves, but not the larger political economy in which households were embedded.  These differential experiences have little to do with the relative \emph{quality} of private and government schools, but with the \emph{legibility} of private providers.

\section*{Mechanisms: How do Preferences Change?\label{section:mechanisms}}

	Here I present several pieces of evidence that are suggestive of why some preferences changed and others did not, as well as how preferences changed. Teachers -- in addition to their role as classroom teachers -- serve as census enumerators, poll booth monitors, and are responsible for organizing major public campaigns such as immunization drives, serving as the front-line functionaries of the Indian state in rural areas \citep{Lipsky2010}.  Voucher winners did not reduce their contact with government school teachers in these core \emph{bureaucratic} functions.  There was also no difference in either objective or subjective quality: the children of vouchers winners did no better academically and voucher parents did not report better treatment by private schools.  Instead, I argue that the primary mechanism driving changing in preferences over the relative importance of the private sector in the provision of services was that access to private schools increased voucher recipients' confidence in the private sector as a permanent actor in their lives. Access to these formerly elite spaces made private schools, and by extension the private sector, more legible.

	\subsection*{Evaluations of Front-Line Functionaries}

		The first mechanism I explore is the relationships households have with schoolteachers and principals.  Earlier, I suggested that exit from government schools could break ties households have with teachers, and remove the discretion teachers have in the provision of public services.  Teachers were present in the lives of households, irrespective of whether they were in the treatment or control groups.  Table \ref{table:summarystatistics} shows that 67 percent of respondents across treatment and control reported experiences with government school teachers working as census enumerators, and 79 percent of respondents reported experience with government school teachers working as election monitors. This is also in a context where the last national census occurred two years before I fielded my survey and the last elections took place the month before I fielded the survey.  Evidence from Bihar suggests that election monitors have influence on turnout and vote choice and have discretion in contexts where voters have low levels of education \citep{Neggers2018}.

		To evaluate whether this relationship had changed as a result of having access to private schools, I repeat the analysis conducted in Figures \ref{fig:partisan} to \ref{fig:valuation} in Figure \ref{fig:monitor} for contact with the front-line functionaries of the Indian educational state.  Despite exiting the state as a service provider, government teachers were still highly visible: voucher winners were just as likely to report an interaction with government school teachers as either a poll booth monitor or census enumerator, and recognize the teacher as both teacher and state representative.

		\begin{figure}[htbp]
			\caption{Non-Teaching Work of Government School Teachers\label{fig:monitor}}
			\centering
			\begin{minipage}{5.5in}
				\includegraphics[width=5.5in, keepaspectratio=true]{Data/output/figures/figuremonitor.pdf}
				\footnotesize
				\emph{Notes:} Each plot presents results of a regression of the dependent variable on the left axis, on enrolling a child in a private school, instrumented by the original voucher offer.  Solid lines represent 95\% confidence intervals and tick marks represent 90\% confidence intervals.  Effect sizes are standardized to be comparable across specifications.  All specifications include district fixed effects.  Robust standard errors are clustered at the school at baseline. Tables \ref{table:appendixelectionmonitors} to \ref{table:appendixcensusenumerators} are the corresponding regression tables for these plots.
			\end{minipage}
		\end{figure}

		Government school teachers in their non-teaching capacity were just as present in the lives of voucher winners, suggesting that exit from schools did not also lead to exit from these relationships.  Individuals believed they derived direct material benefits from political participation.  One respondent believed (incorrectly) that if they did not vote they would be struck off electoral rolls that were used to not only determine eligible voters, but also the beneficiaries of government programs.\footnote{Interview in Nizamabad District, October 2013.}  Individuals vote because they expect to gain direct material benefits from being \emph{seen} to have participated.  In a context where actual and perceived political participation are important for receiving public and private benefits, households continued to participate politically, even after exiting one government service. Exploring these relationships more generally, households that send their children to private schools in the 2011-12 wave of the India Human Development Survey were nine percent more likely to report knowing a government school teacher (Author's own calculations using \citep{Desai2018}).  Teachers and schools appear to play an important role in communities in addition to their roles inside of schools.

	\subsection*{Comfort Rather Than Quality}

		Next, I turn to whether objective or subjective quality was different between students in government and private schools.  If objective or subjective quality was higher in private schools, this could be driving the findings that voucher winners had a preference for private schools.  With regards to objective quality, \cite{Muralidharan2015} find that voucher winning students performed no better after two or four years on tests of Telugu (the local language), or mathematics, and science and social studies after four years, although they had higher scores on tests of English after two and four years, and Hindi after four years.\footnote{\cite{Muralidharan2015} only administered tests of science and social studies or Hindi after four years.}  However, private schools taught more subjects and spent less instructional time on these subjects, suggesting they were more efficient in achieving these scores.  These results are replicated in part in a different context in India by \cite{Dixon2019} who find that a voucher program in Delhi also increased English test scores after four years.

		I explore these results for the downstream sample in Appendix Section \ref{appendix:academics} and find similar results to the larger student population.  On objective but unobservable measures of quality, private schools did not achieve greater academic outcomes than government schools except on English scores, suggesting that more subjective experiences are driving differences in experience.  For subjects such as Telugu and mathematics, students performed no better between treatment and control schools, and for English, students performed better.

		Field observations across government and private schools suggest that the main difference in infrastructural quality between government and private schools was not that private schools had better infrastructure, but that the quality of infrastructure was more consistent between private schools.\footnote{Interview with Government School principal in Kedapa District, November 2013; Interview with Private School Principal in Kedapa District, November 2013; Interview with Government School principal in Medak District, December 2013.}  While some government schools were clean and well kept, others were often in a state of disrepair.  Private schools, however, were consistently of the same quality that was lower than the best government schools but better than the worst.  Therefore, it was not necessarily true that private schools would be of better quality than government schools measured by infrastructure.\footnote{I explore self-reported measures of quality further in Appendix \ref{appendix:quality} by looking at self-reported evaluations of the quality of teachers and school buildings by parents.}

		Although test scores and academic performance on tests were never cited as a reason for entering the voucher lottery, parents might still perceive quality using other markers such as how they or their children were treated in private schools.  In Figure \ref{fig:evaluation} I plot results of questions that ask respondents whether they thought government teachers cared about the well-being of their students and whether they also treated all students equally.  I find no difference between voucher winning and losing households on how they believed teachers treated their children.  These sets of results --- that there were no differences in test scores, or objective quality, and treatment, or subjective quality --- suggests it is not quality that is driving these results.

		\begin{figure}[htbp]
			\caption{Evaluation of School Teachers\label{fig:evaluation}}
			\centering
			\begin{minipage}{5.5in}
				\includegraphics[width=5.5in, keepaspectratio=true]{Data/output/figures/figureevaluation.pdf}
				\footnotesize
				\emph{Notes}: Each plot presents results of a regression of the dependent variable on the left axis, on enrolling a child in a private school, instrumented by the original voucher offer.  Solid lines represent 95\% confidence intervals and tick marks represent 90\% confidence intervals.  Effect sizes are standardized to be comparable across specifications.  All specifications include district fixed effects.  Robust standard errors are clustered at the school at baseline. Tables \ref{table:appendixteacherscare} through \ref{table:appendixschoolteacherqual} are the corresponding regression tables for these plots.
			\end{minipage}
		\end{figure}

		A channel that was made abundantly clear during interviews was the idea of increased comfort with the private sector because of a perceived permanence and legibility of private schools.  For voucher winners, the ability to send their children to private schools for four years, the access to private schools \emph{after} the voucher period was over, and the continued interaction with private schools suggests that a strong mechanism through which respondents became more comfortable with the private sector was through the idea that private schools were a permanent economic actor.

		One role of the vouchers was to make the functioning of the private sector more legible to voucher recipients.  A voucher lottery loser, when asked if they would prefer to hold a government or private sector job, argued that government jobs were more stable because private companies were likely to leave if profits dried up.  When asked about the value of private education, they expressed a similar fear that private schools were likely to abandon their village when they realized there were no profits to be made in low-income rural areas.\footnote{Interview in Medak District, September 2013.}  While a voucher winner also expressed a similar preference for government employment, they cited the benefits of government employment, not the uncertainty of private employment.\footnote{Interview in Nizamabad District, October 2013.}  This suggests that comfort with private service providers and the permanence of a private actor in the lives of respondents has effects beyond merely the direct service provided, instilling confidence with the private sector as a permanent economic actor.  Another voucher lottery loser who sent her three grandchildren to a local Islamic religious school, was concerned that private school fee structures were too complex for her to understand.  On the other hand, the local Islamic religious school provided education to Muslim families for free so she did not have to worry about understanding or paying fees.\footnote{Interview in Visakhapatnam District, September 2013.}

		Vouchers also highlighted that private schools would now become permanent economic actors in the communities in which they opened.  While government schools had a long history in the villages they operated, with some having been built prior to Indian independence in 1947, but most built around the time of independence, private schools on the whole were much newer, with few being more than ten years old at the beginning of the voucher experiment.  Private schools were perceived as ephemeral; one household believed that private schools would disappear when profits dried up as many had before.\footnote{Interview in Medak District, September 2013.}  A voucher recipient was surprised when she was able to keep her daughter in the private school for four years as the school managed to stay open for the entire period.\footnote{Interview in Visakhapatnam District, November 2013.}  The responses from interview subjects was substantiated by the physical \emph{presence} of school buildings.  While government schools all had dedicated buildings and yards that were clearly demarcated and located in the center of most villages, private schools often operated out of houses and apartments that had been converted for the purpose of running a school, giving greater weight to the idea that they could pack up and leave at the first sign of trouble.  While quality did not differ between treatment and control groups, experiences with private schools made them more legible to respondents, both through a better understanding of how they functioned as well as changed beliefs on their permanence as economic actors, allowing for a transfer of this belief to other private services and actors.

\section*{Discussion \& Conclusion\label{section:conclusion}}

	Academic and policy debates around education in India since the early 2000s have focused on two closely related outcomes: increasing enrollment and retention \citep{Banerji2008} and improving test scores \citep{ASER2017}.  The private sector has been seen as a solution to these twin problems as private provision is believed to help the understaffed and overstretched public sector and provide better quality education \citep{Muralidharan2015}.  India has recently introduced a nationwide program similar to the voucher program evaluated here \citep{GovernmentofIndia2009}.  Lost in this debate, however, are important questions on the appropriate role for states and markets in providing public services, and how these then come to affect the relationship between citizens and the state.  The expansion of public schooling has been recognized as an important marker of increased state capacity \citep{Mangla2015a}, while increased private provision is often seen as an indicator of low state-capacity.

	Five years after households entered a private school voucher lottery, I find that households that had access to private schools after winning the lottery hold stronger market-oriented beliefs, particularly with regards to continuing engagement with private schools.  I argue that private schools provided households with a qualitative different experience of the private sector.  However, access to private schools had few effects on political participation as measured through partisan activities, voting, or associational membership suggesting that the role of exit is not to disengage households from the Indian state. The one notable exception is membership in self-help groups that can mobilize individuals into politics \citep{Prillaman2022}.  Households report a greater preference for education services to be provided privately, with some spillovers to other services such as government food rations.  This was expressed through revealed preferences by sending a greater number of children to private schools, even once the school voucher has expired, and report a greater willingness to receive cash transfers instead of in-kind transfers from the Indian government.  The last set of results are particularly large, with households willing to forgo approximately 10 percent of their monthly income to receive a cash transfer instead of in-kind transfers from the government for education and food subsidies.  Access to private services seems to create a class of citizens more comfortable with the idea of ``shopping around'' for education and other services \citep{Jeffery2008} as respondents became more comfortable with private markets, similar to what \cite{Margalit2020} find with access to financial markets in the United Kingdom.  These changes in economic preferences happen through a greater comfort with private schools.  Semi-structured interviews revealed the idea of the permanence and legibility of private schools and the private sector.  While control households revealed an uneasiness with the private sector as an ephemeral actor, voucher winning households had a greater comfort in the idea of private schools not closing or leaving their villages.

	There are a couple of limitations that merit discussion.  First, the population from which this study was drawn was self-selected.  While the access to vouchers was randomized, the population that entered the initial voucher lottery was not.  These are households that are, by revealed preference, more predisposed to using private services and potentially more likely to be able to take advantage of these services once granted access.  If unobservable characteristics that made a more household more likely to enter a lottery for private school vouchers are also correlated with my dependent variables, this would likely make these effects stronger in this subsample relative to the general population. This would raise questions about the generalizability of the results for both policy and theory building. Additionally, only looking at education raises concerns that the effects found here might be a function of the particulars of the education sector and are not generalizable to other services.  Indeed, the strongest results are precisely on continuing with education services.  Interview evidence and results in Figure \ref{fig:valuation} suggest there are spillovers to comfort with the private sector at large, but we should be cautious in drawing too much from these results.
	
	The policy itself is also unique.  While the policy that this paper leverages was implemented by a well regarded NGO in partnership with academic economists, the larger policy has been implemented by the Government of India.  This raises questions about how the effects I find here will translate to a policy implemented by a different entity, similar to questions about the expansion of policies from NGOs to governments \citep{Banerjee2017}.  It is unclear whether this provides an upper or lower bound on more general effects.  Non-compliance in the experiment was high, but no higher than non-compliance in other similar experiments, and the final rate of private school attendance in the control group of 24\% is similar to the national average. The expansion of a similar policy to the entire country \citep{GovernmentofIndia2009}, instead of five districts in one state would suggest that a greater number of households would be subject to similar experiences. These results provide credible interpretation of how the program could change political and economic preferences above and beyond their effects on student academic outcomes and should be considered when thinking about how to evaluate this larger role out.

	Finally, as I was not involved in the original design and execution of the voucher experiment, I was unable to collect baseline data for the outcomes of interest in this study.  This illustrates both the benefits and limitations of ``downstream'' experiments \citep{Baldwin2011}.  While I am able to provide clean causal identification of an important research and policy question at reduced cost to the researcher, the ``downstream'' researcher does not have full control of the data collection and implementation of the experiment.  Nonetheless, I believe the benefits outweigh the costs in many cases and are a valuable avenue to pursue for researchers.  With the increasing proliferation of randomized control trials in economics and other disciplines, political scientists should engage with the potential effects these supposedly non-political interventions have on mass politics \citep{Humphreys2020}.  This paper provides one such way to do so.

	There are important lessons for countries after market-oriented reforms.  Welfare arrangements that include a large number of non-state providers are becoming the norm in both the OECD \citep{Gingrich2011}, and the developing world \citep{Thachil2009, Cammett2011}.  In education specifically, Chile has a long history of using private school vouchers to encourage poor families to attend private schools \citep{Hsieh2006}.  In Kenya, private schools attract the poor with better educational outcomes \citep{Bold2015, Duflo2015a}.  The growing number of schools run by religious organizations in countries such as diverse as India, Mali, and Pakistan can also be seen as a similar manifestation of state exit \citep{Andrabi2006, Bleck2015,Thachil2011}.  If this form of service provision becomes increasingly prominent, it is important for political scientists to understand their effects on a broader range of outcomes.  As a result, these results are also of interest to policy makers.  This paper provides some ways to understand the value individuals place on private and public services, borrowing measures of ``willingness to pay'' for services from economics \citep{Cohen2010}.  More generally, it suggests we should explore both material and interpretive effects seriously.  While the measures in this paper were context dependent and relied on the interplay between qualitative fieldwork to design close-ended quantitative survey questions relevant to the questions of interest \citep{Thachil2017}, we should look to better understand the effects of long-term processes on how individuals view their social world independent of material politics.

	My results suggest we should take the non-material effects of new experiences more seriously when theorizing about political socialization and behavior.  As China and India continue to integrate into the global economy, and households, particularly rural households, that had little experience with the market are given access to an institutionalized market, their expectations of these actors begins to change and have important implications for political attitudes and behaviors.  Beyond private school vouchers, new and sustained experiences with these economic institutions will likely have experiences independent of any material effects they have in increasing or reducing material welfare.  In a context such as rural India, where the economic lives of many households are precarious \citep{Gupta2012,Krishna2010}, these new, stable, experiences have the potential to shape attitudes in politically consequential ways.  Institutional change, such as market-oriented reforms, create their own downstream behavioral and attitudinal effects.  Here I find those effects to be greater comfort with markets in education specifically, and markets in service provision more generally.  While actual political behavior might be unchanged, citizens opinions \emph{are} changing by private service provision.  Policy makers might find it difficult to roll back any reforms towards privatization enacted today as long as these reforms continue to build a mass public more comfortable with the idea of the private sector providing a large number of services.  This is particularly important for India where the program evaluated here was designed to mimic actual government policy.

	The mechanisms I propose suggest that one way that institutional learning occurs is through experiences with services \citep{Pierson2000b}, even if those experiences are sometimes not positive. Here learning occurs through new experiences.  This integrates two literatures in political science that have been kept relatively separate: historical institutional and experimental methods \citep{Steinmo2015}.  Although the intervention and experiment explored here are relatively short term, if the the results hold over a longer period of time or even grow more pronounced, this would help explain how institutional change becomes habituated at an individual level.

	This paper also improves on many existing studies of policy feedback as it is based on a randomly introduced policy experiment thereby providing clean causal estimates of the effect of the private sector.  Reviews of policy feedback have often noted that a common problem plaguing studies of policy feedback have been their inability to make strong causal inferences as they were based on observational data or data at high levels of aggregation \citep{Campbell2012}.  This study addresses both of the shortcomings of that literature.  Here, I suggest that resource effects are perhaps overstated and potentially a result of selection, rather than policy feedback.  I do find strong interpretive effects, suggesting a promising avenue for researchers to explore in greater depth, and propose a new mechanism for them: new experiences change individual's orientations to their social world.  I extend the policy feedback literature to the developing world and also to the idea of the private sector as a politically relevant actor.  Given the rising prominence of non-state actors in service provision \citep{Cammett2011}, it is important for political scientists to take non-state actors and the private sector seriously.  My findings suggest that the private sector can have strong effects on the political attitudes of individuals.

\clearpage

\singlespacing
\bibliography{downstream}
\bibliographystyle{apsr}

\clearpage
\pagenumbering{arabic}% resets `page` counter to 1
\renewcommand*{\thepage}{\thesection\arabic{page}}
\renewcommand\thefigure{\thesection\arabic{figure}}
\renewcommand\thetable{\thesection\arabic{table}}
\setcounter{figure}{0}
\setcounter{table}{0}

\appendix

\doublespacing

\section{Online Appendix}

	\subsection{Interview Sampling\label{appendix:interviews}}

		Sampling for semi-structured interviews was conducted by randomly sampling 24 respondents of the downstream sample with replacement.  In the first-stage of sampling, the sample was stratified into four groups of respondents: voucher recipients that remained in private schools for four years, voucher recipients that did not take-up the voucher, voucher non-recipients that eventually enrolled their children in private schools, voucher non-recipients that never enrolled their children in private schools.  In experimental language, these are compliers (or always-takers), never-takers, always-takers, and compliers (or never-takers) respectively.

		This sampling strategy left six respondents in each group.  In addition, interviews were conducted with the District Education Officer (DEO) of the five districts in which fieldwork was conducted and the city of Hyderabad where I was based for the fieldwork period.  Finally, three longer interviews were conducted with school principals in two private schools (one religions, one secular) and one government school resulting in a total of 33 interviews.  All interviews with lottery participants were conducted in Telugu with the help of an interpreter fluent in English and Telugu.  All interviews with DEOs and school principals were conducted in English as all DEOs and school principals were fluent in English.

		Interviews with lottery participants were conducted in the households of lottery participants, interviews with DEOs were conducted in District offices after regular working hours, and interviews with school principals were conducted on school premises during school hours.

	\clearpage

	\subsection{The Andhra Pradesh School Choice Experiment\label{appendix:apsc}}

		Improving on existing voucher experiments, the Andhra Pradesh School Choice Experiment (APSC) experiment employed a two-stage randomization design.  Villages selected for the project were first randomized into ``treatment'' villages that would receive vouchers and ``control'' villages that would receive no vouchers Once treatment villages were selected, there was a second round of randomization where households within treatment villages were then entered into a voucher lottery.  This created both a child and village level counterfactual where some children within villages received vouchers and others did not, and then some villages had no voucher winning children at all.

		The original experiment operated across five districts of Andhra Pradesh - Nizamabad, Medak, Kedapa, East Godavari, and Visakhapatnam - specifically chosen to account for Andhra Pradesh's cultural and socioeconomic diversity.\footnote{In the recent division of Andhra Pradesh into the two states of Andhra Pradesh and Telengana, Nizamabad and Medak became part of the newly formed Telengana state, while Kedapa, East Godavari, and Visakhapatnam remained in Andhra Pradesh.}  The APSC project was conducted over 180 villages that had at least one recognized private school.\footnote{Recognized private schools are those that have been registered and recognized by the state government.  To receive government recognition, they must meet criteria specified in the Right to Education Act, including a certain pupil-to-teacher ratio, separate boys and girls toilets, a boundary wall to separate the school from other buildings, and a playground for children among other requirements.  These requirements ensure that the quality of the school as measured by physical infrastructure is higher than many unrecognized schools that also operate in the area.}  The initial village level randomization randomized 90 villages into treatment villages, and 90 control villages.  Due to the small size of the program relative to the size of villages, I hold no theoretical expectations for effects at the village, as opposed to household, level.  As a result, I only sampled from treatment villages in which the second stage randomization selected voucher children.  Interested families in all 180 villages were informed of the program and entered into a lottery and informed that entrance into the lottery did not guarantee entry into the program.  The village and household-level randomizations were not conducted publicly.

		A household could only enter a child into the lottery if they were in Upper Kindergarten or Standard 1 at the time of the lottery, ensuring that the child would benefit from five years of private education if they received a voucher.  The voucher provided for fees, books, and school uniforms, but not school lunches or transportation to a \emph{local} private school.\footnote{Government schools in India provide one hot cooked meal a day to students through a Central Government scheme known as the Midday Meal Scheme.  The scheme is often credited with increasing enrollment in government schools \citep{Dreze2001}, although no interviewed households cited the lack of free meals as a barrier to sending their children to private schools if they received a voucher.}  The emphasis on local private schools restricted households to sending children to schools within their village, although villages were purposefully selected to have at least one government recognized private school.  Within treatment villages, 3,097 households applied for a voucher, of which 1,980 (64 percent) were selected by lottery to receive a voucher.  1,210 of the 1,980 (61 percent) households accepted the voucher and enrolled in a private school at the beginning of the project. At the end of five years 980 households (83 percent) remained in private schools.  Table \ref{table:compliance_downstream} provides full details of compliers and non-compliers.

		The voucher was paid directly to the school for all the time that a family sent their child to the school.  Books and materials were provided directly to the households by the schools, not the Azim Premji Foundation \citep[1022]{Muralidharan2015}.  The Government of Andhra Pradesh was not involved in the recruitment, distribution, or administration of the voucher except to grant the Azim Premji Foundation permission to conduct the experiment in the state.  There were no reported cases of local politicians either at the village or state level attempting to claim credit for the program, although it is possible that this happened without the knowledge of the research team or the Azim Premji Foundation.  It is unlikely this happened as the voucher program was small relative to the total school going population in each of these electoral constituencies and the responsibility for education provision rests at the state level in India.

		For a more detailed treatment of the original Andhra Pradesh School Choice Experiment, see \cite{Muralidharan2015}.  The original experiment found that test scores in math and Telugu (the vernacular language of Andhra Pradesh) between students in private schools and government schools were not significantly different.  Two important caveats should be added, however.  While government schools teach an average of three subjects (math, Telugu, and a joint subject of science and social studies called ``environmental studies'' (EVS)) students in private schools study a far greater number of subjects including English, Hindi, separate classes for social studies and science, and computer use.  Accounting for time use within schools, private schools achieve greater ``bang for their buck,'' by achieving the same test scores using less instructional time on the subjects that both types of schools share in common, while attaining higher test scores in subjects, such as Hindi, that government schools do not focus on as much.  Moreover, private schools in the sample spend an average of \rupee3,000 per student per year, while government schools in the sample spend an average of \rupee8,000 per student per year (approximately \$50 and \$130 per student per year respectively).  The second caveat is that there is considerable disruption for students that join private schools not in their vernacular language.\footnote{Private schools in Andhra Pradesh generally fall into two categories, ``Telugu medium'' and ``English medium'' schools.  Telugu medium schools conduct instruction for all classes in Telugu apart from second and third language classes, while English medium schools conduct all instruction in English.}  For students that join Telugu medium private schools, their test scores across all subjects are far better than students in Telugu medium government schools.

	\clearpage

	\subsection{Downstream Sampling\label{appendix:downstream}}

		I should note that as I was not involved the original design and implementation of the voucher lottery and experiment, the outcome variables presented here were not collected at baseline or midline.  I rely on the stronger assumption with randomization that prior beliefs are identical between treatment and control.  As \citeauthor{Duflo2007} \citeyearpar[45]{Duflo2007} argue, ``randomization renders baseline surveys unnecessary, since it ensures the treatment and comparison groups are similar in expectation.''  This assumption does not allow me to test the \emph{relative} changes in prior beliefs, only the \emph{absolute} changes between treatment and control.  I provide some justification in the mechanisms section as to why I think we should interpret the results as ``comfort'' with the private sector rather than hostility to the public sector.

		As we can see from Table \ref{table:balance}, there was covariate balance between treatment and control households in both the full APSC sample and my reduced downstream sub-sample.  As we can see from \ref{table:comparison}, households in my sample were less literate and more households work as agricultural laborers than the district wide averages for the five districts I surveyed in, suggesting my respondents were slightly poorer than district averages.  Households reported far lower rates of unemployement than the district wide average, but this might be a result of different measures of employment between the two surveys, and I surveyed slightly more scheduled caste households and slightly fewer scheduled tribe households.

		Surveyors were instructed to survey the chief decision maker in the household or the person in charge of schooling decisions for children in the household and to return at a later date if either of those two people were not present.  I sampled households from the original APSC sample list, stratifying by district and ensuring a similar balance between treatment and control households as the original intervention.  Surveys lasted between 50 minutes and one and a half hours.  Semi-structured interviews lasted between 30 minutes and two hours.  Simultaneous translation was provided by three of the 11 surveyors who were fluent in English and Telugu.  In this section, I provide further details on attrition in the downstream sample.

		\subsubsection{Attrition\label{appendix:attrition}}

			There could be concern that differential rates of attrition between treatment and control could drive some of the results here.  In Table \ref{table:attrition}, I report the rates of attrition by treatment and control.  I was able to reach 72 percent of households attempted (for a total of 797 households) in the treatment group, and 65 percent of households attempted (for a total of 405 households) in the control group.  The difference in the percentages is statistically significant at conventional levels, suggesting that, \emph{at least for my downstream sample}, there was differential attrition in being able to be reached for this survey.  It is important to note that attrition rates in \cite{Muralidharan2015} are approximately 20 percent and there were no differences between treatment and control in that sample, potentially suggesting that with greater resources, I would have been able to better deal with attrition.
			
			\begin{table}[hbtp!]
				\begin{threeparttable}
					\centering
					\caption{Attrition Between Treatment and Control\label{table:attrition}}
					% latex table generated in R 3.6.2 by xtable 1.8-4 package
% Tue Mar 24 16:31:21 2020
\begin{tabular}{lrrr}
  \toprule
 & Treatment & Control & Difference \\ 
  \midrule
Households Reached (\%) & 28.13 & 35.10 & -6.96 \\ 
  Total Households Reached & 312.00 & 219.00 & 0.02 \\ 
   \bottomrule
\end{tabular}

					\begin{tablenotes}
						\item \emph{Notes:} The third column presents the differences in attrition between treatment and control.  Standard errors are presented below the difference in parentheses and is significant at p = 0.05.
					\end{tablenotes}
				\end{threeparttable}
			\end{table}

			Next, I estimate whether there are differences between attriters and non-attriters in my downstream sample on a number of baseline, three year, and four year covariates by running a model that includes an interaction term between whether a household could not be found and whether they were offered a voucher.  I divide the results into four tables: household characteristics, baseline academic achievement, endline academic achievement, and school choice after three and four years.

			Table \ref{table:attritionhousehold} presents attrition by baseline household characteristics including whether both parents completed primary school in column 1, a household asset index in column 2, whether at least one parent reached grade 10 in column 3, and whether the household is from a scheduled caste or scheduled tribe (SC and ST respectively) in column 4.  The only baseline condition that predicts whether I was unable to reach a household five years after the vouchers were first distributed is whether both parents completed primary school in column 1.  While I was 10 percent less likely to reach a household in which both parents completed primary school, there was no differential attrition between treatment and control households.

			\begin{table}[htb]
				\begin{threeparttable}
					\centering
					\caption{Attrition by Household Characteristics\label{table:attritionhousehold}}
					\input{Data/output/tables/attritionhousehold.tex}
					\begin{tablenotes}
						\item * p < 0.1, ** p < 0.05, *** p < 0.01. Robust standard errors clustered at the school at baseline in parentheses.
						\item \emph{Notes}: All specifications include district fixed effects.
					\end{tablenotes}
				\end{threeparttable}
			\end{table}

			Table \ref{table:attritionacademicbaseline} presents attrition by baseline academic achievement with normalized Telugu scores in column 1 and normalized math scores in column 2.  There are differences in baseline Telugu test scores in households I was unable to reach after five years, although there are no differences between treatment and control for these households.

			\begin{table}[htb]
				\begin{threeparttable}
					\centering
					\caption{Attrition by Baseline Academic Achievement\label{table:attritionacademicbaseline}}
					\input{Data/output/tables/attritionacademicbaseline.tex}
					\begin{tablenotes}
						\item * p < 0.1, ** p < 0.05, *** p < 0.01. Robust standard errors clustered at the school at baseline in parentheses.
						\item \emph{Notes}: All specifications include district fixed effects.
					\end{tablenotes}
				\end{threeparttable}
			\end{table}

			Table \ref{table:attritionacademihindicendline} presents attrition by endline academic achievement in Hindi. It is important to note that endline Hindi tests were only administered to a random subsample of all students in the original voucher experiment, and in this downstream sample, only 646 students were randomly selected from this downstream sample.  Column 1 presents the probability that a child can read Hindi letters, column 2 presents the probability that a child can read Hindi words, column 3 presents the probability that a child can read Hindi sentences, and column 4 the probability that a child can read Hindi paragraphs.  All four specifications show clear differences between treatment and control households that I was unable to reach in Hindi scores.  These results are also consistent with \cite{Muralidharan2015}, where voucher winners had higher test scores in Hindi, a subject that government schools did not teach.

			\begin{table}[htb]
				\begin{threeparttable}
					\centering
					\caption{Attrition by Endline Academic Achievement in Hindi\label{table:attritionacademihindicendline}}
					\input{Data/output/tables/attritionacademichindiendline.tex}
					\begin{tablenotes}
						\item * p < 0.1, ** p < 0.05, *** p < 0.01. Robust standard errors clustered at the school at baseline in parentheses.
						\item \emph{Notes}: All specifications include district fixed effects.
					\end{tablenotes}
				\end{threeparttable}
			\end{table}

			Table \ref{table:attritionacademiothercendline} presents attrition by endline academic achievement in all other subjects.  Column 1 presents normalized test scores for English, science, and social studies (EVS), column 2 for English alone, column 3 for Telugu, column 4 for math, and column 5 for the probability that a student would be present for all the endline tests.  Unsurprisingly, attriters were less likely to be present for all endline tests, and voucher winners had higher test scores in English, a subject government schools did not teach.

			\begin{table}[htb]
				\begin{threeparttable}
					\centering
					\caption{Attrition by Endline Academic Achievement in All Other Subjects\label{table:attritionacademiothercendline}}
					\input{Data/output/tables/attritionacademicotherendline.tex}
					\begin{tablenotes}
						\item * p < 0.1, ** p < 0.05, *** p < 0.01. Robust standard errors clustered at the school at baseline in parentheses.
						\item \emph{Notes}: All specifications include district fixed effects.
					\end{tablenotes}
				\end{threeparttable}
			\end{table}

			Finally, \ref{table:attritionschoolchoice} presents attrition results for a households school choice after four years.  Again, unsurprisingly vouchers winners are more likely to attend both English and Telugu medium private schools.

			\begin{table}[htb]
				\begin{threeparttable}
					\centering
					\caption{Attrition by School Choice\label{table:attritionschoolchoice}}
					\input{Data/output/tables/attritionschoolchoice.tex}
					\begin{tablenotes}
						\item * p < 0.1, ** p < 0.05, *** p < 0.01. Robust standard errors clustered at the school at baseline in parentheses.
						\item \emph{Notes}: All specifications include district fixed effects.
					\end{tablenotes}
				\end{threeparttable}
			\end{table}

			In general, attrition is not predicted by baseline observables, and there is little differential attrition by treatment status.  Households that were hard to find in the original experiment in \cite{Muralidharan2015} were also hard to find in my downstream sample.  While there are some variables that show higher levels of attrition, there are no consistent predictors of attrition, increasing confidence that attrition does not significantly affect differences between treatment and control households.

		\clearpage

		\subsubsection{Comparison to Other Populations}

			In Table \ref{table:comparison}, I provide a comparison of socioeconomic indicators available in both my survey and the Census of India to compare how representative this sample is with the overall population of India.  Despite the school voucher lottery relying on randomized assignment, there exist two prior avenues for selection into treatment.  First, participants had to choose to enter their name into the voucher lottery.  As we can see from Table \ref{table:comparison}, this led to participants that were of lower socioeconomic status than the average household in the same villages, districts, and India as a whole.  This was a feature of the voucher program as the voucher child in the household had to be enrolled in a government school at the time of the lottery, ensuring students of lower socioeconomic status.

			\begin{table}[htbp]
				\begin{threeparttable}
					\caption{Comparison of Sample Socioeconomic Indicators\label{table:comparison}}
					\centering
					\input{Data/output/tables/comparison.tex}
					\begin{tablenotes}
						\item \emph{Source:} 2011 Census of India
						\item \emph{Notes:} Villages surveyed were matched to villages in the 2011 Census of India.
					\end{tablenotes}
				\end{threeparttable}
			\end{table}

			The second avenue for selection emerges from the fact that vouchers winners, upon winning the lottery, had to choose whether to take up the lottery or not, and could drop out at any point, while those that did not win the lottery could choose to send their children to privates schools out of pocket.  As the treatment was over four years, there was a long time for participants to violate assignment status.  Table \ref{table:compliance_downstream} presents compliance in the downstream sample.

	\clearpage

	\subsection{First-Stage Estimator for Instrumental Variable Regression\label{appendix:firststage}}

		As we can see from Table \ref{table:compliance_downstream}, there was about 43 percent non-compliance in the downstream sample in the treatment group after four years and 24 percent non-compliance in the control group after four years (sent their children to private schools of their own accord).  In the regressions that follow in Sections \ref{appendix:fullresults}, I present results using the original voucher lottery as an instrument for four sets of estimators: accepting the voucher, enrolling a child in a private school, having a child in a private school after two years of the voucher experiment, and having a child in a private school after four years of the voucher experiment.  Each estimand presents a different decision by parents that originally entered the lottery.  While the original assignment to treatment and control was random, households could then choose whether they wished to accept the voucher after winning the lottery, and whether they wished to \emph{enroll} their child in a private school after winning the lottery.  These two decisions were only available to households that won the voucher lottery.  The next two sets of decisions, whether the child entered in the voucher lottery was in a private school after two or four years, was also open to non-compliers as households in the control group could independently enroll their children in private schools.

		In Table \ref{table:appendixfirststage} I report the first stage of a  regression of original assignment to treatment and control (``Voucher Winner'') on the four outcomes.  It is important to note that the F-Statistic passes all conventional measures for the strength of an instrument.

		\clearpage

		\begin{landscape}

			\begin{table}[htb]
				\begin{threeparttable}
					\centering
					\caption{First-Stage Regressions\label{table:appendixfirststage}}
					\input{Data/output/tables/firststage}
					\begin{tablenotes}
						\item * p < 0.1, ** p < 0.05, *** p < 0.01. Robust standard errors clustered at the school at baseline in parentheses.
						\item \emph{Notes}: All specifications include district fixed effects.
					\end{tablenotes}
				\end{threeparttable}
			\end{table}

		\end{landscape}

\clearpage

\subsection{Results on Academics\label{appendix:academics}}

In this section I replicate Table VI from \citep{Muralidharan2015} and present the primary results from that paper on educational outcomes for the smaller downstream sample.  Table \ref{table:appendixacademics} presents the effects of winning a voucher on test scores by assignment to treatment, the intention to treat effect.  Table \ref{table:appendixacademicsinstrumented} presents the effects on test scores of enrolling in a private school, instrumented by assignment to treatment, or the treatment on the treated.

Results in this downstream sample are substantively similar to \cite{Muralidharan2015}, although results on Telugu and Math in the second year of the program are significant in this sample while they are not significant in \cite{Muralidharan2015}.  These results provide greater confidence that the downstream sample is representative of the larger population of the original experiment as well the idea that differences in academics are driving the results discussed in the ``\nameref{section:mechanisms}'' Section.

\begin{table}[htbp]
\begin{threeparttable}
\centering
\caption{Impact of Winning a Voucher (Intention to Treat Effect)\label{table:appendixacademics}}
\input{Data/output/tables/academics.tex}
\begin{tablenotes}
\item \emph{Notes}:* p < 0.1, ** p < 0.05, *** p < 0.01. All regressions control for baseline normalized test scores and include district fixed effects. Robust standard errors clustered at the village level in parentheses.  All test scores are normalized relative to the distribution of public school students in control villages by subject and grade. Telugu, math, English, and EVS (science and social studies) test scores are from written end-of-year tests; Hindi test scores are from an individual assessment administered to a representative sample of students. Combined scores are obtained by running a pooled regression across all test scores in each year, with Hindi test score observations weighted up by the inverse of the sampling probability of a student being selected to take the test from the universe of students. Controls include indicators for both parents having completed at least primary school, at least one parent having completed grade 10, and the household being a scheduled caste, as well as a household asset index.
\end{tablenotes}
\end{threeparttable}
\end{table}

\begin{table}[htbp]
\begin{threeparttable}
\centering
\caption{Average Treatment on the Treated (ATT) effect of attending a private school\label{table:appendixacademicsinstrumented}}
\input{Data/output/tables/academicsinstrumented.tex}
\begin{tablenotes}
\item \emph{Notes}:* p < 0.1, ** p < 0.05, *** p < 0.01. All regressions control for baseline normalized test scores and include district fixed effects. Robust standard errors clustered at the village level in parentheses.  All test scores are normalized relative to the distribution of public school students in control villages by subject and grade. Telugu, math, English, and EVS (science and social studies) test scores are from written end-of-year tests; Hindi test scores are from an individual assessment administered to a representative sample of students. Combined scores are obtained by running a pooled regression across all test scores in each year, with Hindi test score observations weighted up by the inverse of the sampling probability of a student being selected to take the test from the universe of students. Controls include indicators for both parents having completed at least primary school, at least one parent having completed grade 10, and the household being a scheduled caste, as well as a household asset index.
\end{tablenotes}
\end{threeparttable}
\end{table}

	\clearpage

	\subsection{Self-Reported Measures of Quality\label{appendix:quality}}

		In the main body, I discuss differential perceptions of quality as a potential mechanism through which parents could evaluate schools and prefer private schools.  Although imperfect, here I provide \emph{self-reported} measures of quality to understand whether parents evaluated private and government schools differently.

		Similar to the main set of results, I construct an index for school quality that aggregates four variables that asked parents to compare their child's school to other schools in the area.  Table \ref{table:appendixindexquality} reports results for the index overall.  Table \ref{table:appendixschoolproblemsolve} reports results for the question ``How effective was your scholarship child's school at solving problems such as discipline, learning levels, or lack of homework with your scholarship child's education?''  Table \ref{table:appendixschoolbuildingqual} reports results for the question ``How do you think that the building grounds and teaching aids in your scholarship child's school compared with those in other schools in the village?''  Table \ref{table:appendixschoolteacherqual} reports results for ``How do you think that the quality of the teachers in your scholarship child's school compared with those in the other schools in the village?''  Finally, Table \ref{table:appendixschooloverallquality} reports results for the question ``Looking at all the factors, how you think your scholarship child's school compared with the schools in the village?''

		All results suggest there are no significant differences between treatment and control households on how they evaluate the quality of their children schools.

		\begin{landscape}
			\begin{table}
				\begin{threeparttable}
					\centering
					\caption{Full Results: School Quality Index\label{table:appendixindexquality}}
					\input{Data/output/tables/index_quality.tex}
					\begin{tablenotes}
						\item * p < 0.1, ** p < 0.05, *** p < 0.01. Robust standard errors clustered at the school at baseline in parentheses.
						\item \emph{Notes}: Each column represents a regression on the School Quality Index.   Columns 1 \& 2 use the original assignment to treatment as the main variable of interest and represent the intention-to-treat (ITT) estimator.  Columns 3 \& 4 instrument having accepted the voucher by the original randomization.  Columns 5 \& 6 instrument enrolling a child in a private school by the original randomization.  Columns 7 \& 8 instrument sending a child to a private school for two years by the original randomization.  Columns 9 \& 10 instrument sending a child to a private school for four years by the original randomization.  All specifications include district fixed effects.
					\end{tablenotes}
				\end{threeparttable}
			\end{table}
		\end{landscape}
	\clearpage

		\begin{landscape}
			\begin{table}
				\begin{threeparttable}
					\centering
					\caption{Full Results: Schools Solve Problems\label{table:appendixschoolproblemsolve}}
					\input{Data/output/tables/school_problem_solve.tex}
					\begin{tablenotes}
						\item * p < 0.1, ** p < 0.05, *** p < 0.01. Robust standard errors clustered at the school at baseline in parentheses.
						\item \emph{Notes}: Each column represents a regression on whether parents consider the school effective at solving problems.   Columns 1 \& 2 use the original assignment to treatment as the main variable of interest and represent the intention-to-treat (ITT) estimator.  Columns 3 \& 4 instrument having accepted the voucher by the original randomization.  Columns 5 \& 6 instrument enrolling a child in a private school by the original randomization.  Columns 7 \& 8 instrument sending a child to a private school for two years by the original randomization.  Columns 9 \& 10 instrument sending a child to a private school for four years by the original randomization.  All specifications include district fixed effects.
					\end{tablenotes}
				\end{threeparttable}
			\end{table}
		\end{landscape}
	\clearpage

		\begin{landscape}
			\begin{table}
				\begin{threeparttable}
					\centering
					\caption{Full Results: School Buildings of High Quality\label{table:appendixschoolbuildingqual}}
					\input{Data/output/tables/school_building_qual.tex}
					\begin{tablenotes}
						\item * p < 0.1, ** p < 0.05, *** p < 0.01. Robust standard errors clustered at the school at baseline in parentheses.
						\item \emph{Notes}: Each column represents a regression on whether parents considered school buildings to be of high quality relative to be nearby schools.   Columns 1 \& 2 use the original assignment to treatment as the main variable of interest and represent the intention-to-treat (ITT) estimator.  Columns 3 \& 4 instrument having accepted the voucher by the original randomization.  Columns 5 \& 6 instrument enrolling a child in a private school by the original randomization.  Columns 7 \& 8 instrument sending a child to a private school for two years by the original randomization.  Columns 9 \& 10 instrument sending a child to a private school for four years by the original randomization.  All specifications include district fixed effects.
					\end{tablenotes}
				\end{threeparttable}
			\end{table}
		\end{landscape}
	\clearpage

	\begin{landscape}
		\begin{table}
			\begin{threeparttable}
				\centering
				\caption{Full Results: School Teachers are High Quality\label{table:appendixschoolteacherqual}}
				\input{Data/output/tables/school_teacher_qual.tex}
				\begin{tablenotes}
					\item * p < 0.1, ** p < 0.05, *** p < 0.01. Robust standard errors clustered at the school at baseline in parentheses.
					\item \emph{Notes}: Each column represents a regression on whether parents considered school teachers to be of high quality relative to nearby schools.   Columns 1 \& 2 use the original assignment to treatment as the main variable of interest and represent the intention-to-treat (ITT) estimator.  Columns 3 \& 4 instrument having accepted the voucher by the original randomization.  Columns 5 \& 6 instrument enrolling a child in a private school by the original randomization.  Columns 7 \& 8 instrument sending a child to a private school for two years by the original randomization.  Columns 9 \& 10 instrument sending a child to a private school for four years by the original randomization.  All specifications include district fixed effects.
				\end{tablenotes}
			\end{threeparttable}
		\end{table}
	\end{landscape}
	\clearpage

	\begin{landscape}
		\begin{table}
			\begin{threeparttable}
				\centering
				\caption{Full Results: School Overall Quality is High\label{table:appendixschooloverallquality}}
				\input{Data/output/tables/school_overall_qual.tex}
				\begin{tablenotes}
					\item * p < 0.1, ** p < 0.05, *** p < 0.01. Robust standard errors clustered at the school at baseline in parentheses.
					\item \emph{Notes}: Each column represents a regression on whether parents considered the school to be of high quality overall relative to nearby schools.   Columns 1 \& 2 use the original assignment to treatment as the main variable of interest and represent the intention-to-treat (ITT) estimator.  Columns 3 \& 4 instrument having accepted the voucher by the original randomization.  Columns 5 \& 6 instrument enrolling a child in a private school by the original randomization.  Columns 7 \& 8 instrument sending a child to a private school for two years by the original randomization.  Columns 9 \& 10 instrument sending a child to a private school for four years by the original randomization.  All specifications include district fixed effects.
				\end{tablenotes}
			\end{threeparttable}
		\end{table}
	\end{landscape}

\clearpage
\pagenumbering{arabic}% resets `page` counter to 1
\renewcommand*{\thepage}{\thesection\arabic{page}}
\renewcommand\thefigure{\thesection\arabic{figure}}
\renewcommand\thetable{\thesection\arabic{table}}
\setcounter{figure}{0}
\setcounter{table}{0}

\section{Supplemental Analysis}

	The following section contains supplemental analysis of interest to readers, but not necessary for full comprehension of the manuscript.

	\subsection{English Language Schools as an Aspirational Mechanism}

			Next, I consider an alternative explanation for the results: aspirations driven by a demand for English language education.  This alternative explanation would suggest that it is not experiential experiences within schools that are driving the results resulting in greater preferences for the private sector, but new aspirations that emerge from having access to English language schools.  When choosing schools, voucher lottery winers were allowed to choose any school available to them within their village, and in some cases one of these options was an English language school.  Other scholars have shown that social aspirations in India are often driven by attainment of English as a symbol of both upward mobility and access to new labor market and social opportunities \citep{Lukose2009}.  The demand for English language education was a common reason given during interviews for why families entered the voucher lottery.  English language education is believed to be an entry to a larger market economy and higher lifetime earnings \citep{Fernandes2008, Kapur2010,Lukose2009, Ohara2012}.\footnote{Jaimie \cite{Bleck2015} finds a similar effect with French language education in Mali, where French speaking children server as linguistic brokers for the rest of the family.  The broader point here is that knowledge of the hegemonic language serves as a form of access to material and symbolic goods.}

			If it is these newly met aspirations that are driving the new economic preferences, then we should see stronger results for households that sent their children to English language schools instead of merely private schools.  To test this potential explanation, I use the original voucher lottery assignment to instrument for whether the household sent their voucher child to an English language school.\footnote{There are, of course, two sources of bias in the choice of an English language school.  The first source of bias, that is corrected through an instrumental variable approach, is one in which non-compliers in the control group send their children to English language schools on their own accord.  The second source of bias with English language schools, however, is the decision of the \emph{type} of school once the household has won a voucher.  This source of bias presupposes that there is a \emph{market} of schools from which consumers can choose their preferred school.  I address this source of bias as a potential alternative explanation below.}  Results for \emph{all} variables used are presented below in Column 6 of Tables \ref{table:appendixpolpart} to \ref{table:appendixteacherstreat} in Appendix Section \ref{appendix:fullresults}.

			\begin{figure}[htbp]
				\caption{English Language Education\label{fig:english}}
				\centering
				\begin{minipage}{5.5in}
					\includegraphics[width=5.5in, keepaspectratio=true]{Data/output/figures/figureenglish.pdf}
					\footnotesize
					\emph{Notes}: Each plot represents results for a regression of the dependent variable on the left axis on attending an English medium school instrumented by winning the voucher lottery.  Solid lines represent the 95\% confidence intervals and tick marks represent 90\% confidence intervals.  All regressions also include district fixed effects.  Robust standard errors are clustered at the school at baseline.
				\end{minipage}
			\end{figure}

			Figure \ref{fig:english} presents the results of a regression that uses assignment to treatment and control to instrument for whether households sent their children to English medium schools.\footnote{This assumes that only private schools are English medium schools, a reasonable assumption at the primary level.}  The results largely mirror earlier results, with small differences in point estimates in the results on preference for the private sector the only difference between the instrumental variable results here and earlier.
			
			From these results, it does not appear that my results are being driven by access to English language schools or changed aspirations from access to English language schools.  The results are largely similar between the two specifications while the coefficients between the two results are substantively similar.

		\clearpage

		\subsection{Variable Definitions\label{appendix:var_def}}

			\paragraph{Partisan Political Participation Index} A summary index of 5 variables: \emph{Member of a Political Party}, \emph{Attended Political Meetings}, \emph{Canvassed for a Political Party}, \emph{Distributed Political Leaflets}.

			\paragraph{Member of a Political Party} A variable that was coded 1 if a respondent answered ``Yes'' to the question ``Are you a member of a political party?'' and 0 otherwise.

			\paragraph{Attended Political Meetings} A variable that was coded 1 if a respondent answered ``Yes'' to the question ``Have you participated in a political meeting or gathering such as an election meeting, procession, or rally over the past year?'' and 0 otherwise.

			\paragraph{Canvassed for a Political Party} A variable that was coded 1 if a respondent answered ``Yes'' to the question ``Have you participated in door to door canvassing in the past year?'' and 0 otherwise.

			\paragraph{Distributed Political Leaflets} A variable that was coded 1 if a respondent answered ``Yes'' to the question ``Have you distributed election leafless or put up posters in the past year?'' and 0 otherwise.

			\paragraph{Non-Partisan Political Participation Index}  A summary index of four variables: \emph{Member of Caste Association}, \emph{Member of Cooperative}, \emph{Member of SHG}, \emph{Gram Sabha}.

			\paragraph{Member of Caste Association} A variable that was coded 1 if a respondent answered ``Yes'' to the question ``Are you a member of any religious/caste organisation or association?'' and 0 otherwise.

			\paragraph{Member of Cooperative or Labor Union} A variable that was coded 1 if a respondent answered ``Yes'' to the question ``Do you belong to any other associations and organisation like cooperatives, farmer's associations, trade unions, welfare organisations, school management committees, or cultural and sports organisations?'' and 0 otherwise.

			\paragraph{Member of SHG} A variable that was coded 1 if a respondent answered ``Yes'' to the question ``Are you a member of a local SelfHelp Group?'' and 0 otherwise.

			\paragraph{Gram Sabha} A variable that was coded 1 if a respondent answers ``Yes'' to the question ``Did you attend your village's last \emph{Gram Sabha} meeting?''

			\paragraph{Private Sector Index} A summary index of three variables: \emph{Private Job}, \emph{Private Services}, and \emph{Private Financing}.

			\paragraph{Private Job} A variable that takes the value 1 if a respondent answers ``Private Job'' to the question, ``If you were looking for a job today, would you prefer a government job, a private sector job, or to be self-employed?'' and 0 otherwise.

			\paragraph{Private Services} A variable that takes the value of 1 if a respondent answers ``Private Body'' to the question, ``If you were seeking health services or education for a family member today, would you prefer the service from a government body or from a private body?'' and 0 otherwise.

			\paragraph{Private Financing} A variable that takes the value of 1 if a respondent answers ``Private actors should both finance and administer these services,'' to the question, ``Which statement about the provision of health care and education do you agree with more?'' and 0 otherwise.

			\paragraph{Voucher Child in Private School} A variable that takes the value of 1 if a respondent reports enrolling their voucher lottery child in a private school \emph{after} the voucher lottery period finished.

			\paragraph{Children in Private Schools} A sum of the number of school-aged children in the household enrolled in private schools.

			\paragraph{Use Private Health Services} A variable that takes a value of 1 if a respondent mentions a private health service in answer to the question, ``If a family member falls sick, where would you take them?''

			\paragraph{Valuation of Public Services Index} A summary index of two variables: \emph{Valuation of Government Schools} and \emph{Valuation of Food Subsidies}.

			\paragraph{Valuation of Government Schools} A variable that took the value the respondent stopped at to the question ``If you were given a choice between receiving an annual education scholarship from the government of \rupee X per year or \rupee X/12 per month that you can spend on your child's education in any way you wish (including private school fees, books, uniform, transport, and private tuition), or being able to send your child to the government school for free (as it currently is), what would you prefer?''  The question began by setting X at \rupee3,000 per year (or \rupee250 per month) and progressed in \rupee500 per year increments to \rupee10,000 per year (or \rupee833 per month).  If the respondent rejected the offer at \rupee10,000 per year, the respondent was asked at what value of scholarship would they be indifferent between government provision or receiving a scholarship, and the variable takes this value.

			\paragraph{Valuation of Food Subsidies} A variable that took the value the respondent stopped at to the question ``Would you prefer a monthly cash transfer from the government of \rupee X INSTEAD of your current monthly grain/fuel entitlements at the Ration Shop?''  The question began by setting X at \rupee200 per month and progressed in \rupee50 increments to \rupee1,000 per month.  If the respondent rejected the offer at \rupee1,000 per month, the respondent was asked at what value of cash transfer would they be indifferent between government provision of rations or receiving a cash transfer and the variable takes this value.

			\paragraph{Teachers in this Village Work as Poll Monitors} A variable that takes the value of 1 if a respondent reports knowing that government school teachers work as election monitors.

			\paragraph{Teachers in this Village Work as Census Enumerators} A variable that takes the value of 1 if a respondent reports knowing that government school teachers work as census enumerators.

			\paragraph{Care about well-being of students} Answer to the question ``Do you think that government teachers care about the well-being of their students?'' from ``Very much care'', ``Somewhat care'', ``Somewhat don't care'', ``Very much don't care''.

			\paragraph{Treat all students equally} Coded as 1 if respondents answer ``Yes'' to question ``Do you think that government school teachers treat all students equally?''

		\clearpage

		\subsection{Full Results Tables\label{appendix:fullresults}}

			In this section, I present the full regression results, including the coefficients on all control variables (Column 2 in Tables \ref{table:appendixpolpart} through \ref{table:appendixteacherstreat}).  In Table \ref{table:appendixsummarystatistics}, I provide summary statistics for control variables not listed in Table \ref{table:summarystatistics} or used in the analysis in Figures \ref{fig:partisan} through \ref{fig:monitor}.

			\input{Data/output/tables/summarystatisticsappendix.tex}

			I also include a series of additional robustness checks in this analysis: using treatment status as an instrument for whether households \emph{accepted} the voucher (Column 3), whether households \emph{enrolled} in a private school (Column 4), whether children \emph{remained} in a private school for the entire voucher experiment (this estimator is the one presented in the results of the main body of the paper) (Column 5), and whether households enrolled their children in English medium schools (Column 6).

			We can also view these tables as results that are robust to alternative measures of school choice (e.g. voucher use and years in private school).  Moreover, comparing these measures reveals that the treatment effects are stronger along the intensive margin (i.e. how \emph{much} does each household use private schools, conditional on any attendance) rather than the extensive margin (i.e. any attendance at a private school).  These results add support to one of my primary mechanisms, the idea of private sector \emph{permanence}, as households that spent \emph{more} time in private schools are more likely to hold stronger market-oriented beliefs.

			\clearpage

				\begin{landscape}

					\begin{table}
						\begin{threeparttable}
							\centering
							\caption{Full Results: Political Participation Index\label{table:appendixpolpart}}
							\input{Data/output/tables/index_part_pol.tex}
							\begin{tablenotes}
								\item * p < 0.1, ** p < 0.05, *** p < 0.01. Robust standard errors clustered at the school at baseline in parentheses.
								\item \emph{Notes}: Each column represents a regression on the Political Participation Index.  Column 5 replicates the first regression plotted in Figure \ref{fig:partisan}.  Columns 1 \& 2 use the original assignment to treatment as the main variable of interest and represent the intention-to-treat (ITT) estimator.  Columns 3 \& 4 instrument having accepted the voucher by the original randomization.  Columns 5 \& 6 instrument enrolling a child in a private school by the original randomization.  Columns 7 \& 8 instrument sending a child to a private school for two years by the original randomization.  Columns 9 \& 10 instrument sending a child to a private school for four years by the original randomization.  All specifications include district fixed effects.
							\end{tablenotes}
						\end{threeparttable}
					\end{table}

					\begin{table}
						\begin{threeparttable}
							\centering
							\caption{Full Results: Member of a Political Party\label{table:appendixpoliticalparty}}
							\input{Data/output/tables/polparty.tex}
							\begin{tablenotes}
								\item * p < 0.1, ** p < 0.05, *** p < 0.01. Robust standard errors clustered at the school at baseline in parentheses.
								\item \emph{Notes}: Each column represents a regression on being a member of a political party.  Column 5 replicates the second regression plotted in Figure \ref{fig:partisan}.  Columns 1 \& 2 use the original assignment to treatment as the main variable of interest and represent the intention-to-treat (ITT) estimator.  Columns 3 \& 4 instrument having accepted the voucher by the original randomization.  Columns 5 \& 6 instrument enrolling a child in a private school by the original randomization.  Columns 7 \& 8 instrument sending a child to a private school for two years by the original randomization.  Columns 9 \& 10 instrument sending a child to a private school for four years by the original randomization.  All specifications include district fixed effects.
							\end{tablenotes}
						\end{threeparttable}
					\end{table}
					\clearpage

					\begin{table}
						\begin{threeparttable}
							\centering
							\caption{Full Results: Attended a Political Meeting\label{table:appendixpoliticalmeeting}}
							\input{Data/output/tables/polmeeting.tex}
							\begin{tablenotes}
								\item * p < 0.1, ** p < 0.05, *** p < 0.01. Robust standard errors clustered at the school at baseline in parentheses.
								\item \emph{Notes}: Each column represents a regression on attending a political meeting.  Column 5 replicates the third regression plotted in Figure \ref{fig:partisan}.  Columns 1 \& 2 use the original assignment to treatment as the main variable of interest and represent the intention-to-treat (ITT) estimator.  Columns 3 \& 4 instrument having accepted the voucher by the original randomization.  Columns 5 \& 6 instrument enrolling a child in a private school by the original randomization.  Columns 7 \& 8 instrument sending a child to a private school for two years by the original randomization.  Columns 9 \& 10 instrument sending a child to a private school for four years by the original randomization.  All specifications include district fixed effects.
							\end{tablenotes}
						\end{threeparttable}
					\end{table}

					\clearpage

					\begin{table}
						\begin{threeparttable}
							\centering
							\caption{Full Results: Canvassed for a Political Party\label{table:appendixpoliticalcanvas}}
							\input{Data/output/tables/polcanvas.tex}
							\begin{tablenotes}
								\item * p < 0.1, ** p < 0.05, *** p < 0.01. Robust standard errors clustered at the school at baseline in parentheses.
								\item \emph{Notes}: Each column represents a regression on canvassing for a political party.  Column 5 replicates the fourth regression plotted in Figure \ref{fig:partisan}.  Columns 1 \& 2 use the original assignment to treatment as the main variable of interest and represent the intention-to-treat (ITT) estimator.  Columns 3 \& 4 instrument having accepted the voucher by the original randomization.  Columns 5 \& 6 instrument enrolling a child in a private school by the original randomization.  Columns 7 \& 8 instrument sending a child to a private school for two years by the original randomization.  Columns 9 \& 10 instrument sending a child to a private school for four years by the original randomization.  All specifications include district fixed effects.
							\end{tablenotes}
						\end{threeparttable}
					\end{table}
					\clearpage

					\begin{table}
						\begin{threeparttable}
							\centering
							\caption{Full Results: Distributed Leaflets for a Political Party\label{table:appendixpoliticalleaflets}}
							\input{Data/output/tables/polleaflets.tex}
							\begin{tablenotes}
								\item * p < 0.1, ** p < 0.05, *** p < 0.01. Robust standard errors clustered at the school at baseline in parentheses.
								\item \emph{Notes}: Each column represents a regression on distributing leaflets for a political party.  Column 5 replicates the fifth regression plotted in Figure \ref{fig:partisan}.  Columns 1 \& 2 use the original assignment to treatment as the main variable of interest and represent the intention-to-treat (ITT) estimator.  Columns 3 \& 4 instrument having accepted the voucher by the original randomization.  Columns 5 \& 6 instrument enrolling a child in a private school by the original randomization.  Columns 7 \& 8 instrument sending a child to a private school for two years by the original randomization.  Columns 9 \& 10 instrument sending a child to a private school for four years by the original randomization.  All specifications include district fixed effects.
							\end{tablenotes}
						\end{threeparttable}
					\end{table}
					\clearpage

				\end{landscape}

				\begin{landscape}

					\begin{table}
						\begin{threeparttable}
							\centering
							\caption{Full Results: Non-Partisan Political Participation Index\label{table:appendixnonpartisanpoliticalparticipation}}
							\input{Data/output/tables/index_locdemoc.tex}
							\begin{tablenotes}
								\item * p < 0.1, ** p < 0.05, *** p < 0.01. Robust standard errors clustered at the school at baseline in parentheses.
								\item \emph{Notes}: Each column represents a regression on the non-partisan political participation index.  Column 5 replicates the first regression plotted in Figure \ref{fig:nonpartisan}.  Columns 1 \& 2 use the original assignment to treatment as the main variable of interest and represent the intention-to-treat (ITT) estimator.  Columns 3 \& 4 instrument having accepted the voucher by the original randomization.  Columns 5 \& 6 instrument enrolling a child in a private school by the original randomization.  Columns 7 \& 8 instrument sending a child to a private school for two years by the original randomization.  Columns 9 \& 10 instrument sending a child to a private school for four years by the original randomization.  All specifications include district fixed effects.
							\end{tablenotes}
						\end{threeparttable}
					\end{table}

					\clearpage

					\begin{table}
						\begin{threeparttable}
							\centering
							\caption{Full Results: Member of a Caste Association\label{table:appendixcasteassociation}}
							\input{Data/output/tables/assccaste}
							\begin{tablenotes}
								\item * p < 0.1, ** p < 0.05, *** p < 0.01. Robust standard errors clustered at the school at baseline in parentheses.
								\item \emph{Notes}: Each column represents a regression on being a member of a caste association.  Column 5 replicates the second regression plotted in Figure \ref{fig:nonpartisan}.  Columns 1 \& 2 use the original assignment to treatment as the main variable of interest and represent the intention-to-treat (ITT) estimator.  Columns 3 \& 4 instrument having accepted the voucher by the original randomization.  Columns 5 \& 6 instrument enrolling a child in a private school by the original randomization.  Columns 7 \& 8 instrument sending a child to a private school for two years by the original randomization.  Columns 9 \& 10 instrument sending a child to a private school for four years by the original randomization.  All specifications include district fixed effects.
							\end{tablenotes}
						\end{threeparttable}
					\end{table}

					\clearpage

					\begin{table}
						\begin{threeparttable}
							\centering
							\caption{Full Results: Member of a Cooperative Association\label{table:appendixcoops}}
							\input{Data/output/tables/assccoops}
							\begin{tablenotes}
								\item * p < 0.1, ** p < 0.05, *** p < 0.01. Robust standard errors clustered at the school at baseline in parentheses.
								\item \emph{Notes}: Each column represents a regression on being a member of a cooperative association.  Column 5 replicates the third regression plotted in Figure \ref{fig:nonpartisan}.  Columns 1 \& 2 use the original assignment to treatment as the main variable of interest and represent the intention-to-treat (ITT) estimator.  Columns 3 \& 4 instrument having accepted the voucher by the original randomization.  Columns 5 \& 6 instrument enrolling a child in a private school by the original randomization.  Columns 7 \& 8 instrument sending a child to a private school for two years by the original randomization.  Columns 9 \& 10 instrument sending a child to a private school for four years by the original randomization.  All specifications include district fixed effects.
							\end{tablenotes}
						\end{threeparttable}
					\end{table}

					\clearpage

					\begin{table}
						\begin{threeparttable}
							\centering
							\caption{Full Results: Member of a Self-Help Group\label{table:appendixshg}}
							\input{Data/output/tables/asscshg}
							\begin{tablenotes}
								\item * p < 0.1, ** p < 0.05, *** p < 0.01. Robust standard errors clustered at the school at baseline in parentheses.
								\item \emph{Notes}: Each column represents a regression on being a member of a self-help group.  Column 5 replicates the fourth regression plotted in Figure \ref{fig:nonpartisan}.  Columns 1 \& 2 use the original assignment to treatment as the main variable of interest and represent the intention-to-treat (ITT) estimator.  Columns 3 \& 4 instrument having accepted the voucher by the original randomization.  Columns 5 \& 6 instrument enrolling a child in a private school by the original randomization.  Columns 7 \& 8 instrument sending a child to a private school for two years by the original randomization.  Columns 9 \& 10 instrument sending a child to a private school for four years by the original randomization.  All specifications include district fixed effects.
							\end{tablenotes}
						\end{threeparttable}
					\end{table}

					\clearpage

					\begin{table}
						\begin{threeparttable}
							\centering
							\caption{Full Results: Attended a Gram Sabha Meeting\label{table:appendixgramsabha}}
							\input{Data/output/tables/gramsabha}
							\begin{tablenotes}
								\item * p < 0.1, ** p < 0.05, *** p < 0.01. Robust standard errors clustered at the school at baseline in parentheses.
								\item \emph{Notes}: Each column represents a regression on attending the meeting of the village Gram Sabha.  Column 5 replicates the fifth regression plotted in Figure \ref{fig:nonpartisan}.  Columns 1 \& 2 use the original assignment to treatment as the main variable of interest and represent the intention-to-treat (ITT) estimator.  Columns 3 \& 4 instrument having accepted the voucher by the original randomization.  Columns 5 \& 6 instrument enrolling a child in a private school by the original randomization.  Columns 7 \& 8 instrument sending a child to a private school for two years by the original randomization.  Columns 9 \& 10 instrument sending a child to a private school for four years by the original randomization.  All specifications include district fixed effects.
							\end{tablenotes}
						\end{threeparttable}
					\end{table}

				\clearpage

				\end{landscape}

				\begin{landscape}

				\begin{table}
					\begin{threeparttable}
						\centering
						\caption{Full Results: Private Services Index\label{table:appendixprivateservices}}
						\input{Data/output/tables/index_privatisation}
						\begin{tablenotes}
							\item * p < 0.1, ** p < 0.05, *** p < 0.01. Robust standard errors clustered at the school at baseline in parentheses.
							\item \emph{Notes}: Each column represents a regression on the private services index.  Column 5 replicates the first regression plotted in Figure \ref{fig:privatisation}.  Columns 1 \& 2 use the original assignment to treatment as the main variable of interest and represent the intention-to-treat (ITT) estimator.  Columns 3 \& 4 instrument having accepted the voucher by the original randomization.  Columns 5 \& 6 instrument enrolling a child in a private school by the original randomization.  Columns 7 \& 8 instrument sending a child to a private school for two years by the original randomization.  Columns 9 \& 10 instrument sending a child to a private school for four years by the original randomization.  All specifications include district fixed effects.
						\end{tablenotes}
					\end{threeparttable}
				\end{table}

				\clearpage

				\begin{table}
					\begin{threeparttable}
						\centering
						\caption{Full Results: Prefer Job in the Private Sector\label{table:appendixprivatesectorjob}}
						\input{Data/output/tables/privjob}
						\begin{tablenotes}
							\item * p < 0.1, ** p < 0.05, *** p < 0.01. Robust standard errors clustered at the school at baseline in parentheses.
							\item \emph{Notes}: Each column represents a regression on a preference for a job in the private sector.  Column 5 replicates the second regression plotted in Figure \ref{fig:privatisation}.  Columns 1 \& 2 use the original assignment to treatment as the main variable of interest and represent the intention-to-treat (ITT) estimator.  Columns 3 \& 4 instrument having accepted the voucher by the original randomization.  Columns 5 \& 6 instrument enrolling a child in a private school by the original randomization.  Columns 7 \& 8 instrument sending a child to a private school for two years by the original randomization.  Columns 9 \& 10 instrument sending a child to a private school for four years by the original randomization.  All specifications include district fixed effects.
						\end{tablenotes}
					\end{threeparttable}
				\end{table}

				\clearpage

				\begin{table}
					\begin{threeparttable}
						\centering
						\caption{Full Results: Prefer the Private Provision of Services\label{table:appendixprivateserviceprovision}}
						\input{Data/output/tables/privservices}
						\begin{tablenotes}
							\item * p < 0.1, ** p < 0.05, *** p < 0.01. Robust standard errors clustered at the school at baseline in parentheses.
							\item \emph{Notes}: Each column represents a regression on a preference for services such as health and education to be provided by the private sector.  Column 5 replicates the third regression plotted in Figure \ref{fig:privatisation}.  Columns 1 \& 2 use the original assignment to treatment as the main variable of interest and represent the intention-to-treat (ITT) estimator.  Columns 3 \& 4 instrument having accepted the voucher by the original randomization.  Columns 5 \& 6 instrument enrolling a child in a private school by the original randomization.  Columns 7 \& 8 instrument sending a child to a private school for two years by the original randomization.  Columns 9 \& 10 instrument sending a child to a private school for four years by the original randomization.  All specifications include district fixed effects.
						\end{tablenotes}
					\end{threeparttable}
				\end{table}
				\clearpage

				\begin{table}
					\begin{threeparttable}
						\centering
						\caption{Full Results: Prefer the Private Financing of Services\label{table:appendixprivateservicefinancing}}
						\input{Data/output/tables/privfinancing}
						\begin{tablenotes}
							\item * p < 0.1, ** p < 0.05, *** p < 0.01. Robust standard errors clustered at the school at baseline in parentheses.
							\item \emph{Notes}: Each column represents a regression on a preference for services such as health and education to be financed by the private sector.  Column 5 replicates the fourth regression plotted in Figure \ref{fig:privatisation}.  Columns 1 \& 2 use the original assignment to treatment as the main variable of interest and represent the intention-to-treat (ITT) estimator.  Columns 3 \& 4 instrument having accepted the voucher by the original randomization.  Columns 5 \& 6 instrument enrolling a child in a private school by the original randomization.  Columns 7 \& 8 instrument sending a child to a private school for two years by the original randomization.  Columns 9 \& 10 instrument sending a child to a private school for four years by the original randomization.  All specifications include district fixed effects.
						\end{tablenotes}
					\end{threeparttable}
				\end{table}
				\clearpage

				\begin{table}
					\begin{threeparttable}
						\centering
						\caption{Full Results: Voucher Child Continued in Private School\label{table:appendixprivateschool}}
						\input{Data/output/tables/privschool1}
						\begin{tablenotes}
							\item * p < 0.1, ** p < 0.05, *** p < 0.01. Robust standard errors clustered at the school at baseline in parentheses.
							\item \emph{Notes}: Each column represents a regression on the voucher child continuing in a private school after the voucher period is over.  Column 5 replicates the fifth regression plotted in Figure \ref{fig:privatisation}.  Columns 1 \& 2 use the original assignment to treatment as the main variable of interest and represent the intention-to-treat (ITT) estimator.  Columns 3 \& 4 instrument having accepted the voucher by the original randomization.  Columns 5 \& 6 instrument enrolling a child in a private school by the original randomization.  Columns 7 \& 8 instrument sending a child to a private school for two years by the original randomization.  Columns 9 \& 10 instrument sending a child to a private school for four years by the original randomization.  All specifications include district fixed effects.
						\end{tablenotes}
					\end{threeparttable}
				\end{table}
				\clearpage

				\begin{table}
					\begin{threeparttable}
						\centering
						\caption{Full Results: Number of Children in Private Schools\label{table:appendixnumberprivateschools}}
						\input{Data/output/tables/numprivschool}
						\begin{tablenotes}
							\item * p < 0.1, ** p < 0.05, *** p < 0.01. Robust standard errors clustered at the school at baseline in parentheses.
							\item \emph{Notes}: Each column represents a regression on the number of children in private school.  Column 5 replicates the sixth regression plotted in Figure \ref{fig:privatisation}.  Columns 1 \& 2 use the original assignment to treatment as the main variable of interest and represent the intention-to-treat (ITT) estimator.  Columns 3 \& 4 instrument having accepted the voucher by the original randomization.  Columns 5 \& 6 instrument enrolling a child in a private school by the original randomization.  Columns 7 \& 8 instrument sending a child to a private school for two years by the original randomization.  Columns 9 \& 10 instrument sending a child to a private school for four years by the original randomization.  All specifications include district fixed effects.
						\end{tablenotes}
					\end{threeparttable}
				\end{table}

				\clearpage

				\begin{table}
					\begin{threeparttable}
						\centering
						\caption{Full Results: Prefer a Private Doctor\label{table:appendixpreferprivatedoctor}}
						\input{Data/output/tables/healthpriv}
						\begin{tablenotes}
							\item * p < 0.1, ** p < 0.05, *** p < 0.01. Robust standard errors clustered at the school at baseline in parentheses.
							\item \emph{Notes}: Each column represents a regression on a preference for a private doctor.  Column 5 replicates the seventh regression plotted in Figure \ref{fig:privatisation}.  Columns 1 \& 2 use the original assignment to treatment as the main variable of interest and represent the intention-to-treat (ITT) estimator.  Columns 3 \& 4 instrument having accepted the voucher by the original randomization.  Columns 5 \& 6 instrument enrolling a child in a private school by the original randomization.  Columns 7 \& 8 instrument sending a child to a private school for two years by the original randomization.  Columns 9 \& 10 instrument sending a child to a private school for four years by the original randomization.  All specifications include district fixed effects.
						\end{tablenotes}
					\end{threeparttable}
				\end{table}

				\end{landscape}

				\clearpage

				\begin{landscape}

				\begin{table}
					\begin{threeparttable}
						\centering
						\caption{Full Results: Valuation of Government Services Index\label{table:appendixvaluation}}
						\input{Data/output/tables/index_valuation}
						\begin{tablenotes}
							\item * p < 0.1, ** p < 0.05, *** p < 0.01. Robust standard errors clustered at the school at baseline in parentheses.
							\item \emph{Notes}: Each column represents a regression on the valuation of government services index.  Column 5 replicates the first regression plotted in Figure \ref{fig:valuation}.  Columns 1 \& 2 use the original assignment to treatment as the main variable of interest and represent the intention-to-treat (ITT) estimator.  Columns 3 \& 4 instrument having accepted the voucher by the original randomization.  Columns 5 \& 6 instrument enrolling a child in a private school by the original randomization.  Columns 7 \& 8 instrument sending a child to a private school for two years by the original randomization.  Columns 9 \& 10 instrument sending a child to a private school for four years by the original randomization.  All specifications include district fixed effects.
						\end{tablenotes}
					\end{threeparttable}
				\end{table}

				\clearpage

				\begin{table}
					\begin{threeparttable}
						\centering
						\caption{Full Results: Valuation of Food Subsidies\label{table:appendixvaluationpds}}
						\input{Data/output/tables/sd_min_amount_pds}
						\begin{tablenotes}
							\item * p < 0.1, ** p < 0.05, *** p < 0.01. Robust standard errors clustered at the school at baseline in parentheses.
							\item \emph{Notes}: Each column represents a regression on the valuation of food subsidies.  Column 5 replicates the second regression plotted in Figure \ref{fig:valuation}.  Columns 1 \& 2 use the original assignment to treatment as the main variable of interest and represent the intention-to-treat (ITT) estimator.  Columns 3 \& 4 instrument having accepted the voucher by the original randomization.  Columns 5 \& 6 instrument enrolling a child in a private school by the original randomization.  Columns 7 \& 8 instrument sending a child to a private school for two years by the original randomization.  Columns 9 \& 10 instrument sending a child to a private school for four years by the original randomization.  All specifications include district fixed effects.
						\end{tablenotes}
					\end{threeparttable}
				\end{table}

				\clearpage

				\begin{table}
					\begin{threeparttable}
						\centering
						\caption{Full Results: Valuation of Government Schools\label{table:appendixvaluationgovschools}}
						\input{Data/output/tables/sd_min_amount_voucher}
						\begin{tablenotes}
							\item * p < 0.1, ** p < 0.05, *** p < 0.01. Robust standard errors clustered at the school at baseline in parentheses.
							\item \emph{Notes}: Each column represents a regression on the valuation of government schools.  Column 5 replicates the third regression plotted in Figure \ref{fig:valuation}.  Columns 1 \& 2 use the original assignment to treatment as the main variable of interest and represent the intention-to-treat (ITT) estimator.  Columns 3 \& 4 instrument having accepted the voucher by the original randomization.  Columns 5 \& 6 instrument enrolling a child in a private school by the original randomization.  Columns 7 \& 8 instrument sending a child to a private school for two years by the original randomization.  Columns 9 \& 10 instrument sending a child to a private school for four years by the original randomization.  All specifications include district fixed effects.
						\end{tablenotes}
					\end{threeparttable}
				\end{table}

				\end{landscape}

				\clearpage

				\begin{landscape}

				\begin{table}
					\begin{threeparttable}
						\centering
						\caption{Full Results: Teachers in this Village Work as Election Monitors\label{table:appendixelectionmonitors}}
						\input{Data/output/tables/pol_gov_school_monitor}
						\begin{tablenotes}
							\item * p < 0.1, ** p < 0.05, *** p < 0.01. Robust standard errors clustered at the school at baseline in parentheses.
							\item \emph{Notes}: Each column represents a regression on whether respondents report that government school teachers work as election monitors.  Column 5 replicates the first regression plotted in Figure \ref{fig:monitor}.  Columns 1 \& 2 use the original assignment to treatment as the main variable of interest and represent the intention-to-treat (ITT) estimator.  Columns 3 \& 4 instrument having accepted the voucher by the original randomization.  Columns 5 \& 6 instrument enrolling a child in a private school by the original randomization.  Columns 7 \& 8 instrument sending a child to a private school for two years by the original randomization.  Columns 9 \& 10 instrument sending a child to a private school for four years by the original randomization.  All specifications include district fixed effects.
						\end{tablenotes}
					\end{threeparttable}
				\end{table}

				\clearpage

				\begin{table}
					\begin{threeparttable}
						\centering
						\caption{Full Results: Teachers in this Village Work as Census Enumerators\label{table:appendixcensusenumerators}}
						\input{Data/output/tables/pol_gov_school_census}
						\begin{tablenotes}
							\item * p < 0.1, ** p < 0.05, *** p < 0.01. Robust standard errors clustered at the school at baseline in parentheses.
							\item \emph{Notes}: Each column represents a regression on whether respondents report that government school teachers work as census enumerators.  Column 5 replicates the second regression plotted in Figure \ref{fig:monitor}.  Columns 1 \& 2 use the original assignment to treatment as the main variable of interest and represent the intention-to-treat (ITT) estimator.  Columns 3 \& 4 instrument having accepted the voucher by the original randomization.  Columns 5 \& 6 instrument enrolling a child in a private school by the original randomization.  Columns 7 \& 8 instrument sending a child to a private school for two years by the original randomization.  Columns 9 \& 10 instrument sending a child to a private school for four years by the original randomization.  All specifications include district fixed effects.
						\end{tablenotes}
					\end{threeparttable}
				\end{table}

				\clearpage

				\end{landscape}

				\begin{landscape}

				\begin{table}
					\begin{threeparttable}
						\centering
						\caption{Full Results: Teachers Care about the Well-Being of Students\label{table:appendixteacherscare}}
						\input{Data/output/tables/per_gov_teacher}
						\begin{tablenotes}
							\item * p < 0.1, ** p < 0.05, *** p < 0.01. Robust standard errors clustered at the school at baseline in parentheses.
							\item \emph{Notes}: Each column represents a regression on whether respondents report that government school teachers care about the well-being of their students.  Column 5 replicates the first regression plotted in Figure \ref{fig:evaluation}.  Columns 1 \& 2 use the original assignment to treatment as the main variable of interest and represent the intention-to-treat (ITT) estimator.  Columns 3 \& 4 instrument having accepted the voucher by the original randomization.  Columns 5 \& 6 instrument enrolling a child in a private school by the original randomization.  Columns 7 \& 8 instrument sending a child to a private school for two years by the original randomization.  Columns 9 \& 10 instrument sending a child to a private school for four years by the original randomization.  All specifications include district fixed effects.
						\end{tablenotes}
					\end{threeparttable}
				\end{table}

				\clearpage

				\begin{table}
					\begin{threeparttable}
						\centering
						\caption{Full Results: Teachers Treat All-Students Equally\label{table:appendixteacherstreat}}
						\input{Data/output/tables/per_treat_gov_teacher}
						\begin{tablenotes}
							\item * p < 0.1, ** p < 0.05, *** p < 0.01. Robust standard errors clustered at the school at baseline in parentheses.
							\item \emph{Notes}: Each column represents a regression on whether respondents report that government school teachers treat all students equally.  Column 5 replicates the second regression plotted in Figure \ref{fig:evaluation}.  Columns 1 \& 2 use the original assignment to treatment as the main variable of interest and represent the intention-to-treat (ITT) estimator.  Columns 3 \& 4 instrument having accepted the voucher by the original randomization.  Columns 5 \& 6 instrument enrolling a child in a private school by the original randomization.  Columns 7 \& 8 instrument sending a child to a private school for two years by the original randomization.  Columns 9 \& 10 instrument sending a child to a private school for four years by the original randomization.  All specifications include district fixed effects.
						\end{tablenotes}
					\end{threeparttable}
				\end{table}

			\end{landscape}

		\subsection*{Voting}

			I did not ask respondents directly what parties they had voted for or intended to vote for for two reasons.  First, due to limited resources, I would be unable to ask this question in a manner that protected respondent confidentiality and the secret ballot while still maintaining a relatively large sample size.  Second, I was concerned about damaging the trust that surveyors and the implementing organization had established with respondents over the previous five years of the intervention and was sensitive to concerns from the implementing organization that wanted to continue working in these villages in the future.  I did, however, ask whether respondents had voted in the most recent elections and whether they intended to vote in upcoming elections.  These results were not included in any of the results presented in the main body of the paper as reported turnout in the sample was high and, at least in India, is a relatively costless activity \citep{Banerjee2011a,Kasara2015a}.  I present those results below.

			I re-run the results from Figure \ref{fig:partisan} and Table \ref{table:appendixpolpart} including voting in the index.  I present results for the index overall and self-reported having voted in the panchayat elections (that had taken place a couple of weeks before going to the field), and self-reported intention to vote for the 2014 \emph{Vidhan Sabha} (state) and \emph{Lok Sabha} (national) elections.

			\begin{landscape}

				\begin{table}
					\begin{threeparttable}
						\centering
						\caption{Full Results: Political Participation Index Including Voting\label{table:appendixpolpartvoting}}
						\input{Data/output/tables/index_voting}
						\begin{tablenotes}
							\item * p < 0.1, ** p < 0.05, *** p < 0.01. Robust standard errors clustered at the school at baseline in parentheses.
							\item \emph{Notes}: Each column represents a regression on whether respondents report that government school teachers treat all students equally.  Column 5 replicates the second regression plotted in Figure \ref{fig:evaluation}.  Columns 1 \& 2 use the original assignment to treatment as the main variable of interest and represent the intention-to-treat (ITT) estimator.  Columns 3 \& 4 instrument having accepted the voucher by the original randomization.  Columns 5 \& 6 instrument enrolling a child in a private school by the original randomization.  Columns 7 \& 8 instrument sending a child to a private school for two years by the original randomization.  Columns 9 \& 10 instrument sending a child to a private school for four years by the original randomization.  All specifications include district fixed effects.
						\end{tablenotes}
					\end{threeparttable}
				\end{table}

				\clearpage

				\begin{table}
					\begin{threeparttable}
						\centering
						\caption{Full Results: Self-Reported Vote in Panchayat Election\label{table:appendixpanchayat}}
						\input{Data/output/tables/votepan}
						\begin{tablenotes}
							\item * p < 0.1, ** p < 0.05, *** p < 0.01. Robust standard errors clustered at the school at baseline in parentheses.
							\item \emph{Notes}: Each column represents a regression on whether respondents report that government school teachers treat all students equally.  Column 5 replicates the second regression plotted in Figure \ref{fig:evaluation}.  Columns 1 \& 2 use the original assignment to treatment as the main variable of interest and represent the intention-to-treat (ITT) estimator.  Columns 3 \& 4 instrument having accepted the voucher by the original randomization.  Columns 5 \& 6 instrument enrolling a child in a private school by the original randomization.  Columns 7 \& 8 instrument sending a child to a private school for two years by the original randomization.  Columns 9 \& 10 instrument sending a child to a private school for four years by the original randomization.  All specifications include district fixed effects.
						\end{tablenotes}
					\end{threeparttable}
				\end{table}

				\clearpage

				\begin{table}
					\begin{threeparttable}
						\centering
						\caption{Full Results: Self-Reported Vote Intention to Vote in \emph{Vidhan Sabha} Election\label{table:appendixvs}}
						\input{Data/output/tables/votemla}
						\begin{tablenotes}
							\item * p < 0.1, ** p < 0.05, *** p < 0.01. Robust standard errors clustered at the school at baseline in parentheses.
							\item \emph{Notes}: Each column represents a regression on whether respondents report that government school teachers treat all students equally.  Column 5 replicates the second regression plotted in Figure \ref{fig:evaluation}.  Columns 1 \& 2 use the original assignment to treatment as the main variable of interest and represent the intention-to-treat (ITT) estimator.  Columns 3 \& 4 instrument having accepted the voucher by the original randomization.  Columns 5 \& 6 instrument enrolling a child in a private school by the original randomization.  Columns 7 \& 8 instrument sending a child to a private school for two years by the original randomization.  Columns 9 \& 10 instrument sending a child to a private school for four years by the original randomization.  All specifications include district fixed effects.
						\end{tablenotes}
					\end{threeparttable}
				\end{table}

				\clearpage

				\begin{table}
					\begin{threeparttable}
						\centering
						\caption{Full Results: Self-Reported Vote Intention to Vote in \emph{Lok Sabha} Election\label{table:appendixls}}
						\input{Data/output/tables/votels}
						\begin{tablenotes}
							\item * p < 0.1, ** p < 0.05, *** p < 0.01. Robust standard errors clustered at the school at baseline in parentheses.
							\item \emph{Notes}: Each column represents a regression on whether respondents report that government school teachers treat all students equally.  Column 5 replicates the second regression plotted in Figure \ref{fig:evaluation}.  Columns 1 \& 2 use the original assignment to treatment as the main variable of interest and represent the intention-to-treat (ITT) estimator.  Columns 3 \& 4 instrument having accepted the voucher by the original randomization.  Columns 5 \& 6 instrument enrolling a child in a private school by the original randomization.  Columns 7 \& 8 instrument sending a child to a private school for two years by the original randomization.  Columns 9 \& 10 instrument sending a child to a private school for four years by the original randomization.  All specifications include district fixed effects.
						\end{tablenotes}
					\end{threeparttable}
				\end{table}

			\end{landscape}
	
\end{document}